\documentclass[openany]{report}
\usepackage[utf8]{inputenc}
\usepackage[T1]{fontenc}
\usepackage[english]{babel}
\usepackage[table,dvipsnames]{xcolor}
\usepackage[pdfpagelabels]{hyperref}
\usepackage{xspace}
\usepackage[title]{appendix}
\usepackage{wrapfig}
\usepackage{graphicx}
\usepackage{subcaption}
\usepackage{csquotes}
\usepackage{longtable,array,booktabs,makecell}
\usepackage{booktabs}
\usepackage{quotes}
\usepackage{awesomebox}
\usepackage{ltablex}

\title{The Mega Man X Compedium}


\newcommand{\x}{\textit{X1}\xspace}
\newcommand{\mhx}{\textit{MHX}\xspace}
\newcommand{\PtIIWarning}{Part \ref{assumptions} preface}
\newcommand{\tabline}{\hskip-0.5pt\vrule width 1pt\hspace{\labelsep}}
\newcommand{\tabdot}{\makebox[0pt]{\textbullet}\tabline}

%\makeatletter
%\@addtoreset{footnote}{section}
%\makeatother

\begin{document}
\pagenumbering{Alph}
\begin{titlepage}
	\maketitle
	\thispagestyle{empty}
\end{titlepage}
\pagenumbering{arabic}

\tableofcontents
\part{Games}
	\section*{Preface}
	This document aims to be a compendium of all the knowledge available of Mega Man X games, from main plot/lore to game mechanics, including trivia, fun facts and even glitches/bugs. For each games in the series information regarding story, characters (although only in a superficial way), weapons, game mechanics will be presented, as well as an analysis of each stage including its secrets and boss (with both story-related and game game-related information). Following games analysis, a more accurate and complete description of each characters will be presented, combining pieces of information coming from different sources, both in-game and from other official materials.

	This document is meant to be accessible and understandable to everyone, from experienced players to new ones, from people who want to learn more about the franchise's lore to ones who want to discover tricks and secret of each games. Moreover it is open for editing to anybody: everyone can suggest additions if something is missing or modifications if any mistakes were made.
	
	Finally, one last remark has to be done. This document is not meant to be a guide of any kind, nor for new players or speedrunners. While some tips for both of these categories will be gives, what is presented here should not be taken as a guide of any kind.

\chapter{Mega Man X}
	%MEGA MAN X.TEX

\textit{Mega Man X} is born as a spin-off game of the classic Mega Man franchise for the Super Famicon/SNES consoles (later also for PC)  between the 1993 and the 1995\cite{wiki:MMX}. The game takes place one hundred years after the main franchise timeline, in a world where humans and special robots capable of having feelings co-exists and where X, one of those robots, fights the evil Sigma which aims to eliminate humans on order to have a world only for robots.

In the year 2005 a remake for PlayStation Portable was done, upgrading to a 3D graphic close to the precedent title, Mega Man X8. This game (named \textit{Maverick Hunter X }or \mhx in short) had the purpose to re-tell the first game's story with some minor changes such as X relationship with Mavericks and Sigma, which weren't present in the original game. Changes also effected level design, such as Light's capsule positions or Sigma stages, completely different from original ones.

\section[Main plot]{Main plot (X)}
In the year 21XX humans live peacefully with new kind of robots: reploids. Reploids (\ref{dict:reploids}) are particular type of robots with the ability to take autonomous decisions as well as having feelings and sentiments\cite{Xcoll1:Manual_X1}. However sometimes  a reploid' electric brain can undergo  malfunctioning making him act dangerously for nearby humans and others reploids . When this happen the reploid is labeled as ``Maverick'' and has to be stopped, in order to be repaired or disposed, by a special organization created with the exact purpose to stop mavericks: the Maverick Haunters. Leader of the Maverick Haunters is Sigma, one of the most advanced reploid of the time. 

The situation takes a turn when Sigma himself go maverick, declaring war against humanity and recruiting in its rank other maverick haunters, both willing to follow him or threatened by Sigma himself. In order to stop the war X, the main protagonist of the story, decides to joining the fight alongside Zero (a close friend of him) on the highway under attack, but is stopped by Vile,
\begin{wrapfigure}{r}{0.4\linewidth}
	\centering
	\includegraphics[width=\linewidth]{figures/X1/Highway_screenshot_4.jpg}
	\caption{Vile capturing X.}
\end{wrapfigure} an ex-haunter released by Sigma from prison, in his ride armor. Only the intervention of Zero himself, which force Vile to flee, allow X to escape safely.

The two then decide to split: while Zero goes to locate the enemy fortress, X has to deal with the threat of Sigma's mavericks. As X defeats all the eight mavericks, acquiring in the process new power-ups and strength, he joins Zero who in the meanwhile has located the Sigma fortress flying over the sea. At the entrance however they are immediately confronted with Vile and his ride armor, which first manages to capture Zero (that challenged him in a one-versus-one fight) and subsequently X too. As Vile laughs at his victory, Zero manages to escape from his cage, latches on the ride armor and detonates himself, destroying it but leaving Vile untouched. Seeing his friend's action give X new energy, allowing him to break free too and face Vile, now vulnerable, defeating him in the process. Before going on however X listens to Zero's ultimate words\cite{wiki:MMX_script}. Zero informs X that his auto-repair system cannot handle all damages taken and that X has a power even greater than his own, which makes him capable of face Sigma (eventually Zero also gives X his own Z-buster if he didn't manage to get his own buster upgrade). After that, Zero dies and X proceeds infiltrating Sigma's fortress and his dangers, including re-facing all the previously-defeated mavericks. At the end X arrives at Sigma's place where Sigma is waiting. At first Sigma acts with arrogance, looking surprised that X has managed to reach him with his only forces and claiming he could destroy him without effort. However he decides to leave the pleasure to his pet Velguarder, as he leaves the scene to assist at the fight. After X successfully manages to destroy Velguarder, Sigma changes his mind a little, understanding why Zero placed his trust in X and claiming X could have been an haunter as strong as he was. After that he proceeds to confront X, but gets defeat and his body destroyed, only his head remaining which merges with a giant wolf-based mechaniloid in the room, giving Sigma a new body to fight X. However X succeeds in destroying the new body as well, effectively getting rid of Sigma for good. As Sigma blames X for destroying his dream of a world only for reploids, Sigma himself and the fortress start to explode, forcing X to teleport to the outside. In the end X ponders about the destruction he helped to generate, on the sacrifices made for the victory, and questioning if choosing to fight was the right choice, meditating if another option was possible. As the flying fortress sinks in the ocean, X realizes that he will have to fight more battles before having the answers he needs. 
After the game's credits, it is then revealed that the next battle will not take long before occurring, since a message form Sigma is played, where the maverick states that his spirit is still intact and waiting for a new body to be built for face X again.

\section{Main Plot (\mhx)}
While being for the major part similar to the original plot, \textit{Maverick Haunter X} takes some divergence regarding characters relationship as well as X's background \cite{wiki:MM_MHX}. One of the main divergence point regard X's story prior to the war's start. Next is a summary of \textit{Maverick Haunter X} differences respect \x.

An in-depth background of the war is given, via the ``Day of $\Sigma$""\ref{App:Day_of_Sigma} OVA, explaining how Sigma started his revolution. Here X's background is also changed, now being already a Maverick Haunter (in contrast with the original, where no information are available) serving in the 17th Elite Unit alongside Zero and commanded by Sigma himself. X's haunter rank is B (in contrast to Zero's SA rank) due to his kindness and pacifist spirit, the only ones suspecting of his true potential being again Zero and Sigma. 

Another character with an expanded background is Vile. Here he aims to destroy both X and Sigma, to create his own world, hence following Sigma plans only due the fact some objectives they have match.

Last main difference regards the final portion of the story\cite{wiki:MM_MHX_script}. In fact, while in the original X and Zero immediately face Vile as they enter Sigma fortress, here X first travel through the fortress, facing previous bosses brought back to life, and only at end he meets with Zero and is stopped by Vile. As the main story, however, Vile captures both of them, resulting in Zero sacrificing and X destroying him. Finally, even Sigma has a different attitude regarding X due already suspecting X's great potential. In fact he first tests X by making him fight Velguarder (in contrast with the original game, where Sigma use Velguarder thinking to be sufficient to defeat X) and then challenges X himself, ending in the same way as the original story.	

\section{Main Characters}
\subsection{X}
X(\ref{char:X}) is the first type of a new kind of sentient robot capable of having feeling and take decision of his own developed by Dr. Thomas Light in the year 20XX. Being provided with great power, Dr. Light needed to ensure X's integrity by running a series of test which would have required thirty years to be completed. Unlikely, Dr. Light lifespan didn't allow himself to live longer to complete all tests, so he sealed X in a capsule capable of run the tests in autonomy.


In the year 21XX X's capsule is found by Dr. Cain\ref{cha:Cain} during an archaeological expedition\cite{X:Manual},\cite{wiki:Cain_journal}. Dr. Cain awakes X and, using Light's designs and X help, develop a new kind of robot later called ``Reploids''. X however remains questioning about his place in the world and the future Dr. Light wanted for him. Despite this, when the evil Sigma starts his war against human kind, X is forced to step up and fight in order to restore peace alongside his friend Zero. 


What told until now refers to the original role of X in the \x game. Nevertheless, as stated before, \textit{Maverick Haunter X} re-write X's story by providing a new background for events prior to Sigma's revolt. In this continuity Dr. light seals X away not for testing his integrity, but rather believing the world not to be ready for X's technology and potential. Here Light is firmly convinced that X has a good spirit and that he will use his power to achieve peace\cite{wiki:MM_MHX_X}. Moreover after his awakening X joins the Maverick Haunters (which do not happen in the original story), serving in the 17th Elite Unit, alongside Zero and under the direct command of Sigma. Despite having great potential, however, X's haunter rank is only B (in contrast with the SA rank of Zero) due is what seems to be his lack of decision during battle. In reality this is caused by his pacifist nature, making him reluctant to fight and preventing him to use his real power\cite{Xcoll1:Manual_X1}. Nevertheless, when the evil Sigma starts his war against human kind, X is forced to step up and fight in order to restore peace together with Zero.

\subsection{Zero}
Fighting alongside X against Sigma is Zero\ref{cha:Zero}. In the original \x game, very few information are give regarding Zero, except being a friend of X and the new leader of the Maverick Haunters\cite{X:Manual}, having the highest rank above all haunters which didn't side with Sigma. 
Different is the situation in the \mhx game, where Zero's relationship with X is explained better, the former being close friend and a mentor for X in the 17th Elite Unit, having an SA haunter rank and working under Sigma. Here it is also stated how Zero repel evil with all his strength, fighting merciless against Maverick, Sigma included\cite{Xcoll1:Manual_X1}.  

In any case, Zero story is the same for both games. He first appears at the end of the highway saving X from Vile's grasp, and asking X to take care of Sigma forces while he locates the enemy hideout. Next he's seen at the entrance of Sigma fortress where he suggests to split up, offering to act as decoy to let X sneak inside. Lastly Zero is seen for the last time at the final battle with Vile, where he sacrifices himself to destroy Vile's armor and allow X to defeat him. Finally with his last words he ask X to go and take down Sigma, eventually giving X his own Z-buster, in case X hasn't upgraded it yet.

\section{Game Mechanics}
\textit{Mega Man X}'s gameplay stays faithful to its original series by being a 2D hybrid between a run' n' gun and a platform, where the main protagonist (X in this case) has to clear different stages in order to unlock the final area of the game. Each stage has its own theme, contains a certain number of items to collect (depending on the stage it could be one ore more) which may require other power-ups to be collected first in order to be obtained (typically bosses weapons). Finally at the end of the stage a boss waits, which has to be defeat in order to clear the stage and which will give X a new weapon based on one of his attacks. As the original series all main stages can be accessed in any order and all bosses presents has a ``weak point'' corresponding to a weapon obtained from defeating one another boss first. The ``weak point'' typically consists in a weapon dealing more damage to the boss, but sometimes others effects can occur, such has stunning him or preventing him to do certain actions.

\x introduces however some new mechanics which will later define the series\cite{wiki:X1_features}:
\begin{itemize}
	\item Dash: X, via the boot parts, can dash and move faster, as well as jump higher via a dash-jump.
	\item Wall-jumping: X can jump onto wall in order to climb them or can slide down to descend slowly. He can also dash-jump off to a wall to cover a greater distance.
	\item Armor parts(\ref{X1:Armor}): By finding Light's capsules (four in total) X can be upgraded unlocking new powers which can help the player during the game.
	\item Sub-Weapon charging(\ref{X1:sub_weapon}): via the buster upgrade, X can not only charge his main X-buster, as the main series, but he can also charge his other sub-weapon to increase damage dealt or change their functionalities.
	\item Heart tanks: beside classical Sub-tank X, eight heart tank are also scattered in various stages (one per each). By picking them up X's energy grows by two, starting from 16 up to a maximum of 32\cite{stratwiki:Heart_tank}.
	\item Stage interactions: although a very limited feature, by clearing certain stages, other ones will be effected and change in some portions.
\end{itemize}

\section{Weapons}\label{X1:sub_weapon}
Here is a list of all sub-weapons available in\textit{ Mega Man X}/ \textit{Mega Man Maverick Haunter X} (\cite{MHX:manual},\cite{wiki:X_weapons}):

\subsection{Shotgun Ice}\label{Shotgun_ice}
Shotgun Ice is the weapon obtained by X after defeating Chill Penguin. It absorbs moisture in the air and fires it in crystallized form. If it hits an enemy or a hard surface, it breaks into 5 pieces which ricochet backward and get destroyed if hit another wall or enemy When charged it creates an Chill Penguin-shaped ice platform (only in \x, while in \mhx is only a sharp sled of ice) in similarly to how the boss creates his ice sculpture. X can stand on the platform which will start moving forward shortly after being created. If X creates the platform and then places himself in the same spot where the platform is creating (due the fact that the creation in not immediate) the platform will slightly push X left or right depending of its position, enabling some glitches such as wall clipping(\ref{X1:misc}, only in \x).

\subsection{Electric Spark}\label{Electric_spark}
Electric spark creates high-pressure voltage within the X-Buster and fires it, for a maximum of three shots at time. If the electric spark hits an hard surface, it splits in half, starting to travel up and down along the surface. The charged version of this weapon differs between the original game and his remake. In X1 upon charged X will release two electric columns in front and behind him and which will move in their respective directions, while in \mhx X will generate electricity in all directions. This weapon is acquired upon defeating Spark Mandrill.

\subsection{Rolling Shield}\label{Rolling_shield}
After beating Armored Armadillo, X will gain the Rolling Shield. By using it X spins energy at high speeds within the Buster and launches it	as an energy shot that rolls along the ground. The generated projectile is about the same size as X (half in \mhx) and will roll following the ground until making contact with an enemy or disappear after a while Only in \mhx the shield will absorb any projectile directed toward it, as well as dispose Mets even while hiding\cite{wiki:Rolling_shield}. Upon contact whit a wall, the shield will ricochet once and disappear up hitting a wall again. When charged X will surround himself with an energy field which eliminates any enemy with less than three hit points that hit it, but will disappear upon contact with enemy with more than that life value. In the original game while the charged shield is active X cannot shoot nor change weapon while in game, requiring the player to change weapon from the pause menu and thus making the shield disappear.

\subsection{Homing Torpedo}\label{Homing_torpedo}
When equipping this sub-weapon X gains the ability to fire up to two (three in \mhx)\cite{wiki:Homing_torpedo} torpedoes capable of tracking enemies. As they pick up speed, they home in on the closest enemy and pursue it. When charged X will release a fan of four (six in \mhx) fish-shaped missiles with greater speed and attack power which home better to enemies. This weapon is obtained after defeating Launch Octopus.

\subsection{Boomerang Cutter}\label{Boomerang_cutter}
The Boomerang Cutter is the weapon obtained upon defeating Boomer Kuwanger. It fires a sharp boomerang made from a special metal. If it does not hit an enemy, it returns to its owner. and if it passes an item on its way back, it picks up the	item and delivers it to its own owner (even bringing dropped life/weapon energy from enemies). Up to three cutters can be on screen at the time\cite{wiki:Boomerang_cutter} and their trajectory will depend on the position X was when he fired them: they will arc upwards if X was standing on the ground, while they will arc downwards if X was in the air. If a boomerang successfully return to X without hitting an enemy, it will replenish the energy used to create it. Upon charged X will release four bigger boomerang spiraling out of him diagonally. In \mhx this has been changed with four boomerang of doubled size which will move back and forward in straight line a few times. 

Finally both Flame Mammoth and Launch Octopus, while not being directly weak to Boomerang Cutter, can be heavily damaged when hit with this weapon, the former loosing his trunk and the latter his tentacles, resulting in both being unable to perform certain moves.

\subsection{Chameleon Sting}\label{Chameleon_sting}
The Chameleon Sting, obtained after beating Sting Chameleon, emits a single straightforward laser which than splits into three directions: forward, up-forward, down-forward. In \mhx it instead directly emits three lasers, which can be angled diagonal-up and diagonal-down and are slightly faster. In both games the charged version makes X flash in various colors of the rainbow making him invincible to any damage (besides insta-kill hazards like pits) for a short amount of time. In the original game X cannot switch to any weapon while the invincibility is active, meanwhile in the remake is free to fire both with the current one and any other weapon\cite{wiki:Chameleon_sting}.

\subsection{Storm Tornado}\label{Storm_tornado}
The Storm Tornado turns the X-buster into a high-power fan that blasts opponents with hard-hitting wind, capable of destroying enemies that stand in its path. It shoots an horizontal tornado which stays on the screen for a short amount of time, and then starts moving in the direction X was facing when shot. Due to his length it can hit enemies multiple times, being an effective way to dispose most of them, especially bigger ones.  The \mhx version has its length halved, allowing to shoot it at a faster rate. When charged the Storm Tornado will create a large vortex covering all the screen in high, surrounding X in the original game while exploding from a shot projectile upon hitting a solid surface in the remake.\cite{wiki:Storm_tornado} It is obtained by defeating Storm Eagle.

\subsection{Fire Wave}\label{Fire_wave}
Fire Wave converts the X-buster into a powerful flamethrower, which deals continuous damage to enemies but that cannot be used underwater. Upon pressing the fire button X will start shooting fire from his X buster at a continuous rate draining energy in the process. Once charged X will fire a fireball which creates a wave of fire upon hitting the ground and moves forward for a while. However in order to charge the weapon X must keep firing, draining energy in the process. This weapon is obtained once Flame Mammoth is defeated. 

\section{First Armor}\label{X1:Armor}
While exploring stages X can run into one of the four capsules hidden by Dr. Light. When entering in contact the capsule will open revealing Dr. Light's hologram which will talk to X and give him one of the armor parts\cite{wiki:First_armor}, explaining in the process how it works.
The First Armor is composed by the following parts:
\begin{itemize}
	\item Foot Parts: Namely ``Emergency Acceleration System''\cite{X:Manual} allow X to dash forward, as well as to perform dash-jumps and wall dash-jumps as well as reducing his hitbox's height a little while dashing. Moreover X can destroy specifics blocks by wall-jumping onto them. The capsule containing the foot parts reside right in the middle of the Snow Mountain Stage, being mandatory to take. In \mhx the capsule is instead at the beginning of Factory stage.
	\begin{figure}[h]
		\centering
		\includegraphics[width=0.5\textwidth]{figures/X1/Armor_foot.jpg}
		\caption{Foot Part capsule location}
	\end{figure}
	
	\item Body Parts: When equipped, X will have all incoming damage reduced by 50\%. It is found in the Forest stage after climbing the wall right before the cave. After defeating RT-55J the capsule will emerge from the ground. Meanwhile \textit{Maverick Haunter X} has moved it in Storm Eagle Stage, requiring the head part to access it.
	
	\item Arm Parts: Allow the X-Buster to be charged up to a third level, and to charge up special weapons. Originally Dr. Light thought this upgrade to be included in the basic X-Buster\cite{X:Manual}, bu he sealed away X when the upgrade was not ready yet, forcing him to turn it into a modification to give to X only later.
	\begin{figure}[h]
		\centering
		\includegraphics[width=0.5\textwidth]{figures/X1/Flame_armor_2.jpg}
		\caption{Buster Part capsule location}
	\end{figure}

	Despite the capsule itself not being mandatory, the game will in any case give the player the buster upgrade in form of the Z-Buster if the player arrives at facing Vile without having found the capsule. While in \x there are no differences between the Arm Parts and the Z-Buster, in \mhx the former allow to shoot the Spiral Shot, which has the same power of the second-level charged shot but is larger and hit multiple times, while the latter deals more damage against bosses that the previous alternatives. In \textit{Meg Man X} the capsule reside in the Factory stage, requiring a precise dash-jump and the Head Parts to break some blocks in the ceiling, while in the remake it is in the same place the Body Parts was in the original game.
	
	\item Head Parts: When equipped X will gain the ability to break specific blocks with an headbutt, as well as avoid damage from falling rocks in the Forest stage. The capsule is found in the Airport stage, hidden behind an obstacle marked with a flammable warning which can be destroyed simply by shooting at it (it is not required the Fire Wave), and in Flame Mammoth stage in the remake, this time requiring the foot part to break blocks hiding it.
	\begin{figure}[h]
		\centering
		\includegraphics[width=0.5\textwidth]{figures/X1/Storm_armor_2.jpg}
		\caption{Head Part capsule location}
	\end{figure}
	
	\item Hadoken. While not being technically a component of the First Armor, rather a hidden bonus, this technique is stored too inside a Light's capsule so it is reported here. It allows X to perform the well-know technique from \textit{Street Fighter} by inputting the button combination $\downarrow$ $\searrow$ $\rightarrow$ (with X facing right) + fire button or using the same combination while X is charging and release the charged shot\cite{RTA_wiki:X1}  AND X is at full health. The projectile deals 32 point of damage\cite{wiki:Hadoken} to all enemies, basically one-shotting every non-shielded enemy, bosses included. The only exception is Sigma's final form, and only in the original \textit{Mega Man X}. 
	
	Both in the original game and its remake the capsule is hidden in Armored Armadillo Stage, see \ref{Hadouken} for more information on how unlock it.
\end{itemize}

\section{Useful Techniques}
Here are listed some useful techniques which the game does not explain but can be helpful while playing\cite{RTA_wiki:X1}: %dash shot, wall dash-jump slope jump, boss charged shots
\begin{itemize}
	\item Higher jumps: If the player keeps the jump button down after jumping, the reached height will be greater than the one of a simple jump.
	
	\item Wall dash-jump: With the dash upgrade, X cannot only dash and dash-jump for greater heights, but can also do it off walls, which can be very useful especially in bosses rooms to jump from a side to the other.
	
	\item Dash shots. If X shoots un uncharged shot while dashing, the projectile will deal two points of damage rather than one. This only apply for one projectile on screen at the time, so shooting multiple projectiles while dashing will give the damage boost only to the first, until it collides or exits from screen (see the file videos/dahs\_shot.mp4 for a practical demonstration).
	
	\item Slope jumps: If X jumps while walking down a slope, he will jump higher than normal.
	
	\item Ladder wall jumps: The top every ladder, where it passes through the floor, can also act as a hole where X can wall jump. In order to do so the player has to hold left or right and press jump to leave the ladder an let X start sliding on the edge of the floor. It is also possible to do this in another manner but only for those ladder were the edge can be reached by X by jumping. To activate the ladder jump X must first start climbing the ladder and then drop down immediately, then he will be able to jump and reach the edge and start wall jumping. If X doesn't grab the ladder first, an invisible wall will stop X from grabbing the ledge and will prevent him to start wall jumping.
	
	\item Staring boss fight with a charged shot: If the player enters a boss room while holding down the fire button and a charged shot ready (any level) and release it while X is out of control, X will it shoot immediately as the boss fight starts.
\end{itemize}

\section{Highway Stage}
The Highway Stage (Central HighWay in \textit{Maverick Haunter X}) is the first stage of the game. X arrives here in order to join the fight against Sigma's forces.
%\begin{wrapfigure}{r}{0.4\linewidth}
%	\centering
%	\includegraphics[width=\linewidth]{figures/X1/Highway_screenshot_1.jpg}
%	\caption{X fighting on the highway.}
%\end{wrapfigure}


 The stage is rather straightforward, having X to travel while eliminating different enemies, while also teaching the player some basic mechanics such as wall-jumping.  The stage can be split into two main sections\cite{stratwiki:HighWay}, the second one having more gaps and pieces of the highway falling as X stands on (recognizable by a darker texture color). In the first section of the highway two sub-bosses are presents in form of \hyperlink{miniboss:Bee_Blader}{Bee Blader}. At the end of the stage X will be attacked by Death Rogumer, which will start sending towards him several \hyperlink{enem:Road_Attackers}{Road Attackers}.

 
 Once some of them have been destroyed, Vile will drop out the airship and start attacking X with his ride armor. Vile is literally invincible, so there is 
 \begin{wrapfigure}[10]{l}{0.4\linewidth}
 	\vspace{-10pt}
 	\centering
 	\includegraphics[width=0.9\linewidth]{figures/X1/Highway_screenshot_2.jpg}
 	\caption{Road Runner dropping on the highway.}
 \end{wrapfigure}
 no point in fighting. Instead as X drop below a certain amount of energy Vile will start shooting an energy projectile toward X and has soon as it connects, X gets trapped and the fight ends, starting the cutscene with Zero saving X. A common speedrun  technique involve having X getting hurt in order to start the boss fight with low health, which will force Vile to shoot the trapping projectile and then run into it, basically skipping the entire fight.  Attention however has to be made, since Vile will not stop attacking with the armor, which may results in the player being defeated.
 
\begin{figure}[htp]
	\centering
	\includegraphics[width=0.6\linewidth]{figures/X1/Highway_screenshot_3.jpg}
	\caption{Vile preparing to face X}
\end{figure}
This stage contains following enemies\cite{wiki:Highway}:
\begin{itemize}
	\item \hyperlink{enem:Ball_De_Voux}{Ball De Voux }
	\item \hyperlink{enem:Bomb_Been}{Bomb Been }
	\item \hyperlink{enem:Crusher}{Crusher }
	\item \hyperlink{enem:Gun_Volt}{Gun Volt}
	\item \hyperlink{enem:Jamminger}{Jamminger}
	\item \hyperlink{enem:Road_Attackers}{Road Attackers }
	\item \hyperlink{enem:Spiky}{Spiky }
	\item \hyperlink{miniboss:Bee_Blader}{Bee Blader }
\end{itemize}

\begin{figure}[htp]
	\centering
	\includegraphics[width=0.6\linewidth]{figures/X1/Full_map.png}
	\caption{Full map with Bosses and their locations}
\end{figure}
 
 
 
\section{Ocean}
The \textit{Ocean stage} (later renamed Subterranean Base in \mhx) is where Launch Octopus hides himself. Here X starts from the shoreline to immerse short after the beginning of the level. Once underwater X will jump higher than normal which, combined with the dash-jump and the beginning of the level being placed on top of a cliff, making the player capable of skipping a good portion of the first part of the underwater scenario\cite{stratwiki:Ocean}. Various fish-based enemies are present, but main difficulties of the sub-bosses the stage has, being three plus an optional one. In the first part main difficulties are given by spiked pit which X has to jump and from where Sea Attacker will spawn in group of three, interrupting X's jump and making him fall into the pit, and from the two Angler sub bosses, the second one being particular dangerous, since the floor has two spiked pits and the Angler will keep using his vacuum attack to push/pull X into them. Here the best strategy consists in stay at the leftmost platform and keep shooting, dashing to the right once it start pulling and shooting while it pushes X. Other than that Anglers will also shoot X four serpent-shaped harpoon which will move horizontally and then fall onto X when they're above him. Finally Anglers will rarely shoot a beam of light from their lamp, which can also be destroyed. After passing the second sub-boss the second part of the stage begins. Here spiked gaps are rarer (although still present) and whirlpools will start to appear at regular intervals and in specific positions. Those whirlpools can be used to propel X up to the ocean's surface. Proceeding in the stage bombs will start dropping from above, fired by a Cruiziler enemy. X can either climb a whirlpool to get onto it and destroy its core, making it sink and stop it from shooting bombs, or avoiding it and proceed in the level. Once passed falling bombs, X will find himself into an arena filled with sand. Here an Utuboros will start attacking him, rising from the sand, than swimming for a while and then immerse again in the sand. It is invincible for most of its body (which X can stand on), only the head and tail being vulnerable. Although its weakness is considered to be the Boomerang Cutter, due to lack of invincibility frame a single Storm Tornado fired behind its head will destroy him in one shot\cite{wiki:Utuboros}. After this other sub-boss X will have go on only for a little before finding himself in front of the boss door.
Following enemies are present in the stage\cite{wiki:Ocean}:
\begin{itemize}
	\item \hyperlink{enem:Amenhopper}{Amenhopper}
	\item \hyperlink{miniboss:Anglerge}{Anglerge}
	\item \hyperlink{enem:Gulpfer}{Gulpfer }
	\item \hyperlink{enem:Cruiziler}{Cruiziler}
	\item \hyperlink{enem:Mega_Tortoise}{Mega Tortoise }
	\item \hyperlink{enem:Sea_Attacker}{Sea Attacker}
	\item \hyperlink{enem:Sky_Claw}{Sky Claw }
	\item \hyperlink{miniboss:Utuboros}{Utuboros}
\end{itemize}

\subsection{Heart Tank}
\begin{figure}[h]
	\centering
	\begin{subfigure}{0.4\textwidth}
		\centering
		\includegraphics[width=\linewidth]{figures/X1/Octopus_heart_1.jpg}
		\caption{}
	\end{subfigure}
	\begin{subfigure}{0.4\textwidth}
		\centering
		\includegraphics[width=\linewidth]{figures/X1/Octopus_heart_2.jpg}
		\caption{}
	\end{subfigure}\\
	\begin{subfigure}{0.4\textwidth}
		\centering
		\includegraphics[width=\linewidth]{figures/X1/Octopus_heart_3.jpg}
		\caption{}
	\end{subfigure}
	\caption{(a)Via whirlpool the player can reach the Cruizer on ocean's surface,(b) When destroyed the ship will fall down, breaking the ocean floor and opening a new path,(c) One destroyed the Utuboros inside the cave, a room will open with the Heart Tank inside}
\end{figure}
This stage only has an heart tank has its collectible. In order to get it the player musts first destroy the Cruizer by reaching the ocean's surface via a whirlpool,climb on it and then shoot at its blue core, avoiding the Sky Claws which the ship will spawn. Once destroyed the Cruizer will sink and will destroy the ocean floor, revealing an hidden portion with a large room filled with spiked gaps on the ground. Here another Utuboros (technically the first since the other reside later in the level) must be defeated, in order to open the door which leads to an Heart Tank. After that X must exit from where he entered and climb the wall where the ship fall in order to return to the main path.

\subsection{Launch Octopus}
Launch Octopus, the ``\textit{General of the Deep Sea}''\cite{book:MMX_Complete_art} was a Maverick Haunter of the 6th fleet armada before joining Sigma's rebellion and setting his base in the dept of the Ocean, in order to attack marine cities with his army of mechaniloids and to cut off shipping routes. In \x he joins Sigma due the fact he shares the dream of a world only for reploids, having always had doubt on protecting humans, while in \mhx he is pictured military tactician which want to achieve beauty in combat, considering himself an unappreciated artist of underwater fighting. Only Sigma understand his art, making Launch Octopus side with him\cite{wiki:MM_MHX_script}.

\begin{figure}[h]
	\centering
	\begin{subfigure}{0.5\textwidth}
		\centering
		\includegraphics[width=\linewidth]{figures/X1/Octopus_missile.jpg}
		\caption{}
	\end{subfigure}
	\begin{subfigure}{0.49\textwidth}
		\centering
		\includegraphics[width=\linewidth]{figures/X1/Octopus_piranha.jpg}
		\caption{}
	\end{subfigure}\\
%	\begin{subfigure}{0.5\textwidth}
%		\centering
%		\includegraphics[width=\linewidth]{figures/X1/Octopus_vortex.jpg}
%		\caption{}
%	\end{subfigure}
%	\begin{subfigure}{0.5\textwidth}
	%		\centering
	%		\includegraphics[width=\linewidth]{figures/X1/Octopus_leech.jpg}
	%		\caption{}
	%	\end{subfigure}
	\caption{Launch Octopus attack: (a) Homing Torpedo,(b) Charged Homing Torpedo.}
\end{figure}

In battle Octopus has two main techniques: Homing torpedo and Energy Drain. In reality he has a third technique, which consists in firing what can be considered a charged version of his Homing Torpedo. Both his types of missiles can be destroyed by shooting at them. Launch Octopus will always start the boss fight with a barrage of homing missile, to than switch to one of the other two weapons he has. Just like X, Octopus can fire a charged version of his homing missiles, which are piranha-shaped and homes much faster than their counterpart. Finally he has his Energy Drain attack. When performing this Octopus will first swim on top of the arena, either left or right, and then start spinning to create a whirlpool to suck X in. If he succeed he will start draining X energy to replenish his own, prolonging the fight. Octopus main weakness is the Rolling Shield, which deals him extra damage but can be hard to hit, due Octopus high mobility and the shield colliding with his missiles instead. However Octopus can be also heavily damaged by the Boomerang Cutter, which in three hits will sever his tentacles blocking him to perform his Energy Drain attack. 

Officially\cite{wayback:X_resources} Launch Octopus is 232 cm tall and 182 Kg heavy, in contrast with in-game information\cite{wiki:Launch_Octopus} which show a weight value smaller than the actual one (see fig \ref{Octopus_specs}).

One Launch Octopus has been defeated X will gain the Homing Torpedo(\ref{Homing_torpedo}) weapon, as well as cause a flood in the Forest Stage, making water appear near the beginning of the stage.
\begin{figure}[htp]
	\centering
	\includegraphics[width=0.6\linewidth]{figures/X1/Launch_octopus_specs.png}
	\caption{Launch Octopus specifications according to X1}
	\label{Octopus_specs}
\end{figure}

\section{Snow Mountain}
\textit{Snow Mountain}(\textit{Abandoned Missile Base in \mhx}) is probably the first stage most people go when starting the game, due to its low level of danger it contains\cite{stratwiki:Snow_mountain}, as for having inside the most useful upgrade X get can get, the Foot Parts. The stage, as the name states, takes place in a snow-covered mountain which X has to climb to get to the boss. The stage is divided into three main areas. The first area is where X starts and consists in a path that brings him inside a cave through a series of enemies. Once inside the cave (the second area) X has first to climb it by following a zig-zag path while also avoiding enemies which drop from higher level, and then to surpass some iced platform separated by pits until he reaches the end of the cave. In the third area X is again on the outside. Here he can take a Ride Armor and proceed for a while. With the armor X can also destroy the igloo he finds and that would otherwise start spawning enemies. Passed the first igloo a choice can be made: to keep the Ride Armor and enter the small cave which comes next, where it is request to jump some gaps using the armor while also face another enemy Ride Armor, or use the armor to reach the top of the cave (jumping out of the armor to gain extra lift), basically skipping the lower portion. After that section X meets a tall wall which the armor cannot surpass, forcing him to leave it behind. Here he has to following a path full of slopes, gaps and Snow Shooters, which will throw snowballs at him and that become bigger the longer they roll on the snowy ground. After passing these enemies, the player will reach the boss door.

The stage contains following enemies\cite{wiki:Snow_mountain}:
\begin{itemize}
	\item \hyperlink{enem:Armor_Soldier}{Armor Soldier} (with Ride Armor)
	\item \hyperlink{enem:Axe_Max}{Axe Max}
	\item \hyperlink{enem:Bat_Bone}{Bat Bone}
	\item \hyperlink{enem:Bomb_Been}{Bomb Been}
	\item \hyperlink{enem:Flammingle}{Flammingle}
	\item \hyperlink{enem:Jamminger}{Jamminger }
	\item \hyperlink{enem:Ray_Bit}{Ray Bit}
	\item \hyperlink{enem:Snow_Shooter}{Snow Shooter}
	\item \hyperlink{enem:Spiky}{Spiky}
	\item \hyperlink{enem:Tombot}{Tombot}
\end{itemize}
\subsection{Foot Parts}\label{X:Foot_Parts}
This stage contains the Armor capsule for the Foot Parts, which allows X to dash, moving faster, perform dash-jumps and wall dash-jumps and alse reducing a little the height of this hitbox. The capsule is found on directly path after having climbed the first part of the cave. Since it is right in the middle of the path it is not avoidable, making it the only mandatory capsule of the entire series.
	\begin{figure}[h]
	\centering
	\includegraphics[width=0.5\textwidth]{figures/X1/Armor_foot.jpg}
	\caption{Foot Part capsule location}
\end{figure}
\subsection{Heart Tank}\label{Penguin:heart_tank}
The Heart Tank of this stage is hidden in the last igloo X meet before facing the section with snow balls. To reach it X has to jump using the Ride Armor. Right after the first igloo there is a light blue column, right before the cave with the enemy Ride Armor. From the top of the column the player has to jump with the armor and, once reached the maximum height, jump out of it in order to reach the wall which brings to the upper path. Here two other igloo are found which can be destroyed, now that the armor is gone, by using the Fire Wave weapon. In particular the last one contains the searched Heart Tank.

\subsection{Chill Penguin}\label{boss:Chill_Penguin}
The ``\textit{Glacial Emperor}''\cite{book:MMX_Complete_art}), better known as Chill Penguin was a reploid of the 13th Polar Division, specifically designed for operating in polar environments also tanks to his small-size body. However due to the long permanence in the South Pole, away from civilization, Chill Penguin grown tired of his mission, seeking a way to get away from there. Sigma's rebellion give him exactly what he needed to get away, allowing him to escape and join Sigma in his fight. There are however some differences in how his story is told between \x and \textit{Maverick Haunter X}. In the former, after Sigma starts his revolt, Chill Penguin autonomously eliminated his own commander and escaped to join Sigma while also returning to the civilization\cite{Xcoll1:Manual_X1}. Once arrived Chill Penguin seized a mountain base in order to crush nearby cities using avalanches. Meanwhile in the remake it is Sigma himself which took away Chill Penguin from the South Pole by making him join the 17th division together with X and Zero, even before the revolution's beginning\cite{MHX:manual}. Later, while facing X, Chill Penguin revealed that, beside leaving the South Pole, he also joined the revolt due the payment Sigma gave him\cite{wiki:MMX_script}. Another minor difference with the original version is that, in the remake, Chill Penguin seized an abandoned missile base, hinting his purpose was to use it to attack the cities. He's known to be a belligerent, rowdy (sometimes even warped) individual, while also having trouble with his partner Flame Mammoth (\ref{boss:Flame_mammoth}) due his jealously of Mammoth size and the latter attitude of bullying smaller reploids.


\begin{figure}[h]
	\centering
	\begin{subfigure}{0.5\textwidth}
		\centering
		\includegraphics[width=\linewidth]{figures/X1/Chill_shot.jpg}
		\caption{}
	\end{subfigure}
	\begin{subfigure}{0.49\textwidth}
		\centering
		\includegraphics[width=\linewidth]{figures/X1/Chill_frozen.jpg}
		\caption{}
	\end{subfigure}\\
	\begin{subfigure}{0.5\textwidth}
		\centering
		\includegraphics[width=\linewidth]{figures/X1/Chill_shatter.jpg}
		\caption{}
	\end{subfigure}
	\begin{subfigure}{0.49\textwidth}
		\centering
		\includegraphics[width=\linewidth]{figures/X1/Chill_slide.jpg}
		\caption{}
	\end{subfigure}\\
	\begin{subfigure}{0.4\textwidth}
		\centering
		\includegraphics[width=\linewidth]{figures/X1/Chill_blizzard.jpg}
		\caption{}
		
	\end{subfigure}
	%	\begin{minipage}{0.49\textwidth}
	%		\centering
	%		\includegraphics[width=\linewidth]{figures/X1/Chill_burn.jpg}
	%	\end{minipage}
	\caption{Chill Penguin's attack: (a) Shotgun Ice, (b) X frozen near the two ice statues, (c) Chill Penguin's shotgun ice stopped by his own statues, (d) Chill Penguin slide attack and (e) blizzard attack.}
	
\end{figure}
Chill Penguin is often considered the easiest boss of the game and the first one a player must face due the simplicity of avoiding his attacks, also thanks to the Foot Parts the stage give to the player. He has four main attack which will perform at random\cite{wiki:Chill_Penguin}. When he use his Shotgun Ice, Chill Penguin will shot four frozen projectiles which travel in a straight line, although some of them  can fall off and slide on the ground, which shatter upon contact with a wall but can nullify X shot if the two met. Other times instead of his projectiles he will emit snow which will create two penguin-shaped ice sculpture and, if X is in range, will also freeze him in place. Sculptures act both as obstacles (X takes damage if he makes contact with them) and as cover since both X and Chill Penguin projectiles will be stopped by them.However X shots can damage and eventually break them. In some occasion Chill Penguin will leap in the air and grab the hook on the ceiling, unleashing a blizzard which will push X and the statues (if present) against a wall. When the sculpture hit a wall they shatter. Finally he can also slide on the floor for a while, turning immediately in case he hits a wall. While in this state he is invincible, but will also get rid the sculpture he creates.  As it is possible to see, Chill Penguins attacks only cover the low part of the arena, meaning the best strategy to fight him consist in staying on the higher part via continuous wall-jumps, only dropping to hit him when possible. Attention however must be made, since sometimes Chill Penguin will perform a high jump toward X position which can hit him even if he's  in the high corner of the room. Beside that he can be hit at any times, even when grabbing the hook, beside when sliding. He's weakness is the Fire Wave, which will burn and stun him for a while, but it has no other particular effect.


After defeating him X will gain the Shotgun Ice(\ref{Shotgun_ice}) weapon as well as freezing the lava in the Factory Stage, considerably reducing the danger level of the stage.

Chill Penguin is 163 cm tall and 108 Kg heavy, despite the information screen on the game report different(\ref{Penguin_specs}).

\begin{figure}[h]
	\centering
	\includegraphics[width=0.5\linewidth]{figures/X1/Chill_penguin_specs.jpg}
	\caption{Chill Penguin specifications according to X1.}
	\label{Penguin_specs}
\end{figure}

\section{Gallery}
The Gallery Stage (or Energy Mines Ruins in \mhx) is the stage where controlled by Armored Armadillo. Peculiarity of this level are its mine cart which X can stand and, as soon as he does it, they start moving following the track, increasing their speed as they keep going and destroying all enemies that enter in contact with it. Hoever the player must be careful, as all the carts end their run into pits, bringing X down with them if he doesn't jump off at the right moment. This stage is also famous for containing the secret Light's capsule which teach X the Hadouken technique, as well as being the best place to farm health capsule and lives.

The stage itself is rather simple. Immediately at the beginning a mine cart is present which brings the player forward in the level while also taking care of enemies, but that end its ride when it comes across a series of gaps, one of which it cannot jump and falls into. For that moment the player must have dismount from the cart. After this sequence only few small gaps and some enemies separate X from the next section, which begin when he has to fall off a long pit in the ground. Here as he touches the floor a Mole Borer will destroy the left wall and start chasing X. It is invincible and its roller can insta-kill X if he touches it. Here only two solutions are possible: the first one is to escape to the right, continuing in the level; the second one is, as soon as X reaches the ground, wall jump immediately to the left before the Mole Borer breaks the wall and then jump down, following it from behind (while also gaining acces to the sub tank). In both cases the player has to go right to continue, followed by the mechaniloid (which will eventually crash and explode) or not. After this section another mine cart waits, which will bring the player deeper in the level and, again, end its ride into a pit. Just like the before after this part another large hole in the ground is present and X has to fall off to proceed, meeting another Mole Borer once he lands. This time however he will land behind it, avoiding another chase and, on the other hand, allow the mechanilod to create a passage for the player, until it crashes and explodes again (the player may also want to destroy it since while going it also destroys the access to the heart tank) . Here there is the final cart X has to take to proceed. This mine cart brings X through many enemies until it reaches an opening on the mountain and flies off over a huge gap, impossible for X to jump. Once in the air the player must jump off the cart in order to land on the other side of the pit (or grab the wall if high enough), where the boss door is. While riding in the final parts a group of mechanical birds enemies will start spawning and fly alongside X. Since they fly the cart cannot destroy them, but X will have to, since they can mess up the final jump by hitting him and making him fall into the pit.

Following enemies are present in the stage\cite{wiki:Gallery}
\begin{itemize}
	\item \hyperlink{enem:Bat_Bone}{Bat Bone} 
	\item \hyperlink{enem:Batton_M-501}{Batton M-501} 
	\item \hyperlink{enem:Dig_Labour}{Dig Labour} 
	\item \hyperlink{enem:Flammingle}{Flammingle} 
	\item \hyperlink{enem:Metal_Wing}{Metal Wing} 
	\item \hyperlink{enem:Metall_C-15}{Metall C-15} 
	\item \hyperlink{miniboss:Mole Borer}{Mole Borer}
	\item \hyperlink{enem:Spiky}{Spiky}
\end{itemize}

\subsection{Sub Tank}
The Sub Tank is located where the first Mole Borer spawn. In order to get it the player has, once arrived to the firs big gap in the ground, jump into it and then immediately wall-jump to the left, in order to let the Mole Borer break the wall and go right passing under X. Once it is passed, the player can safely jump off and go left to find the sub tank where the Mole Borer originally was. 

\subsection{Heart Tank}
The heart tank is located near where the second Mole Borer is found. Once the X jumps off the gap he has to chase the mechaniloid and destroy it in time, since as he proceed it destroy the walls which allow to access the Heart Tank. To do this the Fire Wave weapon is suggested, since it deals massive amount of damage to it in a short span. 

\subsection{Hadouken}\label{Hadouken}
The Hadouken is the last upgrade X can get before facing finals stages. The capsule containing it can be unlocked only of the player has managed to collect every other upgrade in the game: eight Heart Tanks, four Sub-Tank, four armor capsule and all the weapon from bosses. Next the player has to travel trough the gallery stage and reach the top of the ledge above the boss door at least three time after defeating Armored Armadillo, all of them at full health. At the fourth time, near the usual health drop, there will also be the secret capsule.

The best way to make the capsule spawn is as follow: As the stage begin, the player has to travel the level (by walking or on the mine cart) unit it reaches the zone where a single Batton M-501 drops from the roof (near the first slope). Here he should farm lives by continuing killing that enemy (which has a increased drop rate for them)  and making it respawn (it is suggest to use a charged Rolling Shield as it destroy it in one hit). After having accumulated enough lives (at leas five) the player can proceed in the level until the last mine cart is reached, making sure to be at full health by this point. Here X has first to release a charged Sting Chameleon and immediately after ride the cart until over the pit, where he should jump off and grab the ledge over the boss door, reaching the top of if. As X reaches it, he has to jump into the pit, loosing a life and respawning at the last checkpoint (before the second Mole Borer). From here the player has to repeat this procedure for other three times until, at the  fourth one, the capsule will be there. In the remake the capsule does not require multiple travel through the stage, being available immediately.

Once opened the capsule will reveal Dr. Light wearing robe similar to Ryu's one from the Street Fighter series, which will ask X to step into the capsule to teach him the technique. In the Japanese  script of the game\cite{wordpress:X_japanese_script} Light states that he trained under the nearby waterfall a lot to learn it, and that he will now teach it to X who can learn it, as stated again by Light himself in the \mhx remake during the same occasion, due to him having a nearly-human soul.

\subsection{Armored Armadillo}\label{boss:Armored_Armadillo}
Armored Armadillo was the commander in charge of the 8th armored force, as well as a soldier loyal to his superior, Sigma. Once the latter started his revolt, Armored Armadillo blindly followed his orders, occupying a mine to extract raw materials from which Sigma aims to create weapons for his plans.

Faithful to his name and title (``\textit{Armored Warrior}''\cite{book:MMX_Complete_art}) Armored Armadillo fights by using his impenetrable armor both as a shield to deflect X shots and as a weapon to crush him while rolling. When guarding Armadillo is invincible to attacks X performs and, moreover, he can also absorb and re-release charged shot which X fired at him. His main attack is the Rolling Shield, which see him turning into a ball and start bouncing off the walls of the arena while also being immune to any attack. Finally he can sometimes take out the cannon hidden in his forehead and start shooting a sequence of rapid bullet in a straight line. The main difficulty in facing Armored Armadillo is given mainly by the few time he leaves open himself to hits before turning invincible and start attacking again. To solve this problem his weak point comes in play. Armored Armadillo is in fact extremely weak to the Electric Spark since not only this weapon deals him more damage than others but it also can remove his plating. Once deprived of his armor, Armadillo won't be able to guard himself anymore and will also lost his invincibility while rolling, basically making the fight much easier for the player to manage.

Upon beating Armored Armadillo X gains the Rolling Shield weapon(\ref{Rolling_shield}), but no other effects take place in other stages.

According information give, Armored Armadillo is 194 cm tall and 232 kg heavy, while in-game information report him slightly smaller and lighter(\ref{Armadillo_specs}).

\begin{figure}[h]
	\centering
	\includegraphics[width=0.5\linewidth]{figures/X1/Armored_armadillo_specs.png}
	\caption{Armored Armadillo specifications according to X1.}
	\label{Armadillo_specs}
\end{figure}
\section{Factory} 
The \textit{Factory Stage}(\textit{Prototype Weapons Plant} in \mhx)  is probably one of the most difficult stage among the eight, due to the high number of danger X can met and that can kill him instantly. The level takes place inside a factory which has been designated to weapons production, as the numerous conveyor belts, presses, and molten metal suggest and that X must face before reaching the boss. However if X manages to defeat Chill Penguin before facing this level, he will found the metal frozen solid, reducing considerably the danger level of the area, as he can stands on it whether he falls from a conveyor belt.
\begin{figure}[h]
	\centering
	\begin{subfigure}{0.4\textwidth}
		\centering
		\includegraphics[width=\linewidth]{figures/X1/Flame_frozen.jpg}
		\caption{}
	\end{subfigure}
%	\begin{subfigure}{0.49\textwidth}
%		\centering
%		\includegraphics[width=\linewidth]{figures/X1/Flame_molten.jpg}
%		\caption{}
%	\end{subfigure}\\
	\caption{Factory Stage before (a) and after (b)  defeating Chill Penguin}
\end{figure}

The stage can be divided into four parts. At the beginning two sets of conveyor belt are present over molten metal which will try to obstacle his movement, together with  \hyperlink{enem:Scrap_Robo}{Scrap Robos} which will keep spawning on them and that will try attack X by crawling at him or by shooting lasers. At the end of this section X has to jump off a hole into the ground which brings him deeper into the factory. The next section is where all the level power-up are concentrated. I consists in a single big room, again on top of molten metal, without conveyor belts but with multiple platform at different heights from which enemies will attack X throwing down their pickaxes. The main danger here is given by the trajectory of the pickaxes since, being parabolic, allow them to be throw safely from upper levels (eventually even from off-screen) without giving X the chance to fire back until he reaches the same platform the enemy stands on. Reached the exit of this part on the top right of the room, another section with conveyor belts and Scrap Robo is present, this time with the addition of presses which insta-kill X if he gets caught. Finally the last part of the stage sees X walking on pipes. Here dangers comes from molter iron dripping from pipes, although it only damages X,  \hyperlink{enem:Rolling_Gabyoall}{Rolling Gabyoalls} which move along pipes while also flipping up and down and finally \hyperlink{enem:Hoganmer}{Hoganmers}, enemies which can bi hit only when attacking (as they lower their shield) or by attacking them from behind. In this section some ladders are present, but their only purpose it to create side-paths to skip enemies, but their utility is questionable. At the end of this section there is the boss door which leads to Flame Mammoth.

These enemies appears in this stage\cite{wiki:Factory}:
\begin{itemize}
	\item \hyperlink{enem:Dig_Labour}{Dig Labour} 
	\item \hyperlink{enem:Hoganmer}{Hoganmer}
	\item \hyperlink{enem:Metall_C-15}{Metall C-15}
	\item \hyperlink{enem:Scrap_Robo}{Scrap Robo}
	\item \hyperlink{enem:Sky_Claw}{Sky Claw}
	\item \hyperlink{enem:Rolling_Gabyoall}{Rolling Gabyoall}
\end{itemize}

\subsection{Arm Parts}
At the entrance of the big room the player will notice some blocks on the roof which can be broke with the Head parts. In order to do so the player will have to dash-jump to the left from the very end of the first platform in order to make X destroy rightmost block which will allow him to wall-jump on the remaining one. Once X has started wall jumping the player has to keep going, in order to climb while destroying remaining blocks and then to access the capsule with the Arm Parts. It is a very precise jump, so many tries can be needed. Furthermore it is possible that X can destroy the block but not starting to wall jump, falling on the ground. This complicates the situation, since the jump now is even more difficult, but still doable, requiring enough precision to make X land on the remaining block at sufficient height to start a wall jump. If for some reasons the player destroy the second block too the climb became impossible and the player has to kill himself to reset the area.
\begin{figure}[h]
	\centering
	\begin{subfigure}{0.4\textwidth}
		\centering
		\includegraphics[width=\linewidth]{figures/X1/Flame_armor_1.jpg}
		\caption{}
	\end{subfigure}
	\begin{subfigure}{0.5\textwidth}
		\centering
		\includegraphics[width=\linewidth]{figures/X1/Flame_armor_2.jpg}
		\caption{}
	\end{subfigure}\\
	\caption{Buster upgrade position: a dash-jump is required from (a) to to start wall-jumping and break the ceiling to reach (b).}
\end{figure}

\subsection{Sub Tank}
In the big room, while going to the top-right brings to the exits, going to the top-left corner will bring the player to the sub tank. To obtain it the player has first to reach the highest platform (the one with a life up) and then dash-jump to the left to reach the wall and climb it. While climbing X will reach a part of the wall made of block breakable with the Foot Parts. Behind these blocks there is a small path with the sub tank in it.
\begin{figure}[h]
	\centering
	\includegraphics[width=0.35\textwidth]{figures/X1/Flame_tank.jpg}
	\caption{Flame Mammoth's Sub Tank}
\end{figure}



\subsection{Heart Tank}
As the previous power-ups, the Heart Tank too is ``hidden'' in the big room. In truth it isn't hidden at all, since it is in plain sight at the bottom-right corner of the room, floating on top of the molten iron. There is no way X can get it with the iron in the molten state, so the only way to get it is to defeat Chill Penguin first, freezing the ground and allowing X to walk on it to reach the Heart Tank safely.

\begin{figure}[h]
	\centering
%	\begin{subfigure}{0.4\textwidth}
%		\centering
%		\includegraphics[width=\linewidth]{figures/X1/Flame_heart_1.jpg}
%		\caption{}
%	\end{subfigure}
	\begin{subfigure}{0.5\textwidth}
		\centering
		\includegraphics[width=\linewidth]{figures/X1/Flame_heart_2.jpg}
		\caption{}
	\end{subfigure}\\
	\caption{Flame Mammoth's Heart Tank location. To get it defeating Chill Penguin is mandatory.}
\end{figure}

\subsection{Flame Mammoth}\label{boss:Flame_mammoth}
Flame Mammoth, the ``\textit{Blazing Oil Tank}''\cite{book:MMX_Complete_art} was the captain of the 4t Land battalion, with base in the Middle East, before joining Sigma and being labeled as Maverick. His big size and firepower have always been his pride, making him arrogant and cocky, eventually even leading to enjoy bullying/humiliating smaller reploids (which as \cite{wayback:X_resources} states, he even hates), Chill Penguin(\ref{boss:Chill_Penguin})\cite{wiki:Flame_mammoth} included. He's only desire was to show his full power and use it an go into a violent rampage and destroy everything he wants, and Sigma's rebellion was the perfect way to achieve it. However due to being an extremely arrogant and hated leader, no one of his old battalion followed him\cite{MHX:manual}.

The battle against Mammoth takes place in a large arena on top of a conveyor belt Mammoth himself commands. He has three main attacks: When he use his ``Oiling'' he launches a blob of oil (stored in his stomach\cite{wayback:X_resources}) from his trunk. The oil itself does not harm X but can set up a trap when combined with another of Mammoth's attack, the Fire Wave. With this attack, Flame Mammoth shoots several fireballs from his buster towards X which fall onto the ground. If a fireball makes contact with the oil, it bursts into pillar of fire. Finally Mammoth can use his ``Jump Press'' attack, which sees him leaping toward X in order to crush him. Once he lands, a shock-wave is produced which stuns X making him fall onto the ground, if he was on the ground when Mammoth landed. Occasionally Flame Mammoth will also trumpet inverting the direction the conveyor is facing. 

\begin{figure}[h]
	\centering
	\begin{subfigure}{0.4\textwidth}
		\centering
		\includegraphics[width=\linewidth]{figures/X1/Mammoth_oil.jpg}
		\caption{}
	\end{subfigure}
	\begin{subfigure}{0.4\textwidth}
		\centering
		\includegraphics[width=\linewidth]{figures/X1/Mammoth_fire.jpg}
		\caption{}
	\end{subfigure}\\
	\begin{subfigure}{0.5\textwidth}
		\centering
		\includegraphics[width=\linewidth]{figures/X1/Mammoth_oil_fire.jpg}
		\caption{}
	\end{subfigure}
	\begin{subfigure}{0.3\textwidth}
		\centering
		\includegraphics[width=\linewidth]{figures/X1/Mammoth_trunk.jpg}
		\caption{}
	\end{subfigure}
	\end{figure}

	\begin{figure}
	\ContinuedFloat
	\centering
	\begin{subfigure}{0.4\textwidth}
		\centering
		\includegraphics[width=\linewidth]{figures/X1/Mammoth_press_1.jpg}
		\caption{}
	\end{subfigure}
	\begin{subfigure}{0.5\textwidth}
		\centering
		\includegraphics[width=\linewidth]{figures/X1/Mammoth_press_2.jpg}
		\caption{}
	\end{subfigure}
	\caption{Flame Mammoth's attack: (a) Oiling, (b) Fire Wave, (c) Oil ignited by the fire, (d) Trumpet, (e),(f) Jump Press }
\end{figure}

Battling Flame Mammoth can be tricky, but is not impossible. While his oil and Fire Wave attacks are rather simply to dodge, the real threat reside in his press attack, which he will use often and, moreover, he will perform from off-screen. The arena in fact is much wider than all other, not fitting into a single screen. This can lead to the boss to attack from outside the field of view of the player, that has  to act quickly to dodge incoming attacks, such as Mammoth leaping onto X. Furthermore, as stated previously, his jump attack also creates a tremor which stun X if on the ground, so avoid his landing may not be sufficient. Flame Mammoth weakness is the Storm Tornado which however just deals more damage to him and can dispel his flame attack, but it has no other effect on Mammoth himself. What instead has an effect on Flame Mammoth is the boomerang cutter: three hits will cut his trunk, preventing him to spit oil and to reverse the belt direction.

By beating Flame Mammoth X gain the Fire Wave (\ref{Fire_wave}) and, by consequence, the access to Chill Penguin's heart tank (see \ref{Penguin:heart_tank}). However beating this stage has no interaction with others stages.

According to data available, Flame Mammoth is 321 cm tall and 327 Kg heavy (slightly heavier than what portrayed in game).

\begin{figure}[h]
	\centering
	\includegraphics[width=0.5\linewidth]{figures/X1/Flame_mammoth_specs.jpg}
	\caption{Flame Mammoth specifications according to X1.}
\end{figure}
\section{Airport}
In the \textit{Airport stage} (or \textit{New Type Airport} in the remake) X starts from the ground and has to climb up the airport structures in order to reach the Death Rogurmer as it leaves, in order to destroy it. 

The stage focus mainly on platforming above bottomless pit and on moving platforms, as it takes place in the sky. Various enemies are also placed on some of this platform in order to obstacle X while jumping. X starts on the ground and has immediately to climb using some moving platforms to reach the airport's roof, while some \hyperlink{enem:Sky_Claw}{Sky Claws} try to grub him and make him fall into the pit. After reaching the roof, X can either continue along it, fighting more enemies while going, or can destroy the lift cannon at the beginning , and use the platform it leaves behind to reach the top of the nearby tower, where X can enter after destroying the window and go through, skipping below enemies. Next is a platforming section with a series of platform which move up and down in an alternate way. Furthermore on some of them \hyperlink{enem:Flamer}{Flamers} are positioned, which have a good horizontal range and must destroyed in order for X to step on the platform. However they are positioned every other platform, meaning the player can skip them, by dash-jumping from the platform X stands when it's at the peak (meaning the next one with the enemy is lower) and landing two platforms ahead, in an empty one. Following this part, there is a pretty straightforward section where X has to defeat more enemies in order to proceed. Finally the last part begins with a series of three platforms above a bottomless pit which will proceed to fall as X steps on them, giving the player few time to jump onto the following one. Passed this X will reach the Death Rogumer itself and, after passing its two cannons, X will stand in front of the boss door. On the far right of the ship, before entering the boss, a weapon tank and a health tank are present onto the ship wing.

The stage has the following enemies\cite{wiki:Airport}:
\begin{itemize}
	\item \hyperlink{enem:Ball_De_Voux}{Ball De Voux }
	\item \hyperlink{enem:Flamer}{Flamer}
	\item \hyperlink{enem:Gun_Volt}{Gun Volt}
	\item \hyperlink{enem:Hoganmer}{Hoganmer}
	\item \hyperlink{enem:Lift Cannon}{Lift Cannon}
	\item \hyperlink{enem:Metall_C-15}{Metall C-15}
	\item \hyperlink{enem:Sky_Claw}{Sky Claw}
	\item \hyperlink{veichle:Death Rogumer}{Death Rogumer}'s cannons
\end{itemize}

\subsection{Heart Tank}
Right at the beginning of the stage, if when the rising platform reaches its top X jumps left instead of right will land onto a platform right above the starting point but not accessible from below. On this platform is where the Heart Tank is.
\begin{figure}[h]
	\centering
	\begin{subfigure}{0.4\linewidth}
		\centering
		\includegraphics[width=\linewidth]{figures/X1/Storm_heart_1.jpg}
		\caption{}
	\end{subfigure}
	\begin{subfigure}{0.4\linewidth}
		\centering
		\includegraphics[width=\linewidth]{figures/X1/Storm_heart_2.jpg}
		\caption{}
	\end{subfigure}
	\caption{Airport's Heart Tank location. A dash-jump from (a) to the left bring the player to the Heart's location (b).}
\end{figure}

\subsection{Sub Tank}
After reaching the Airport roof the player will find a 	\hyperlink{enem:Lift_cannon}{Lift Cannon}, a cannon on top of a platform which will rise and lower as the cannon spins. Once destroyed the cannon the platform will fall down, only to rise again as X stands on it, to bring him near a window X can destroy and enter the tower. Here a \hyperlink{enem:Gun_Volt}{Gun Volt} is present and, as X destroys it, all windows will shatter, opening and exit on the opposite side X has entered. At the end of the tower is where the Sub Tank is. 
 
 \begin{figure}[h]
 	\centering
 	\begin{subfigure}{0.4\linewidth}
 		\centering
 		\includegraphics[width=\linewidth]{figures/X1/Storm_tank_1.jpg}
 		\caption{}
 	\end{subfigure}
 	\begin{subfigure}{0.4\linewidth}
 		\centering
 		\includegraphics[width=\linewidth]{figures/X1/Storm_tank_2.jpg}
 		\caption{}
 	\end{subfigure}
 	\caption{Airport's Sub Tank location. From the top of the platform X has to break the window (a) and enter the tower. At the end (b) is where the Sub Tank is.}
 \end{figure}
 
\subsection{Head Parts} 
 While traveling the stage the player may notice that in some locations there are canisters bearing a flammable mark onto them. While it is logical to think the Fire Wave weapon is necessary to detonate them, this is not true and X buster's charged shots can work as well, only requiring few more shots. While most of them only hide health tanks or life up, the one near the pylon after the sequence with moving platforms hides a secret path which leads to the Head Parts. 
 
 \begin{figure}[h]
 	\centering
 	\begin{subfigure}{0.4\linewidth}
 		\centering
 		\includegraphics[width=\linewidth]{figures/X1/Storm_armor_1.jpg}
 		\caption{}
 	\end{subfigure}
 	\begin{subfigure}{0.4\linewidth}
 		\centering
 		\includegraphics[width=\linewidth]{figures/X1/Storm_armor_2.jpg}
 		\caption{}
 	\end{subfigure}
 	\caption{Head part location: by destroying the fuel tanks the capsule is on the right.}
 \end{figure}
 
 \subsection{Storm Eagle}\label{boss:Storm_Eagle}
 Storm Eagle was the taciturn, calm, careful strategist and leader of the Maverick Hunters' 7th Airborne Unit\cite{wiki:Storm_Eagle}. Although not talking much and being difficult to approach, Storm Eagle was a very popular leader among his men\cite{MHX:manual}. Loyal to his job as Maverick Haunter, when the rebellion broke out Eagle's first reaction was to haunt down and challenge Sigma. However Sigma was no match for him, and Storm Eagle was forced to surrender. It not clear however if him joining Sigma was mostly due simply to an self-preservation instinct (as the original game and the collection hint\cite{Xcoll1:Manual_X1}) or due his pride, which forced him to follow Sigma even if reluctant (hinted in remake). In both cases Storm Eagle ended up controlling the Death Rogumer, a new type of aerial battleship (the same one X meets at the end of the Highway Stage from which Vile descends).
 
 
 \begin{figure}[h]
 	\centering
 	\begin{subfigure}{0.3\linewidth}
 		\centering
 		\includegraphics[width=\linewidth]{figures/X1/Eagle_egg_1.jpg}
 		\caption{}
 	\end{subfigure}
 	\begin{subfigure}{0.35\linewidth}
 		\centering
 		\includegraphics[width=\linewidth]{figures/X1/Eagle_egg_2.jpg}
 		\caption{}
 	\end{subfigure}\\
	 \begin{subfigure}{0.4\linewidth}
	 	\centering
	 	\includegraphics[width=\linewidth]{figures/X1/Eagle_push.jpg}
	 	\caption{}
	 \end{subfigure}
	 \begin{subfigure}{0.4\linewidth}
	 	\centering
	 	\includegraphics[width=\linewidth]{figures/X1/Eagle_tornado.jpg}
	 	\caption{}
	 \end{subfigure}\\
 		 \begin{subfigure}{0.4\linewidth}
 		\centering
 		\includegraphics[width=\linewidth]{figures/X1/Eagle_dive.jpg}
 		\caption{}
 	\end{subfigure}
 	\caption{Storm Eagle attacks: (a),(b) Bird Summon, (c) Gust, (d) Storm Tornado and (e) Dive.}
 \end{figure}
 
 The difficulty of the battle against Storm Eagle can vary a lot depending on whether X has already obtained the Foot Parts(\ref{X:Foot_Parts}). This is due the fact that most of Storm Eagle's attacks (two out of four) revolve around pushing X off the arena (which is only a long platform onto of a pit with no walls) and the dash completely nullify these attacks. In particular these two attack are his Gust\cite{wiki:Storm_Eagle} and Storm Tornado. The former sees Eagle flapping his wings to create a rush of air which push X away at a slow rate and simply walking can prevent X from falling, if he has enough space left, while the latter generates a horizontal tornado which doesn't deal any damage but push X much faster and, without dashing through, can be hard to counter. Next Storm Eagle has his Diving attack, that makes him rise out of the screen to dive-bomb diagonally onto X several times before returning on the ground. X can either focus on dodging Eagle, which should be rather simple as the arena is enough tall and wide to see Eagle diving in advance, or can even try to hit him, as he's not invincible while performing this attack. Finally Storm Eagle has his Bird Summon attack, that consists in him shooting out an egg from his beak while hovering mid-air and, if the egg hits the ground, it hatches into four mini bird-like robots which fly towards X. Storm Eagle's weakness is the Chameleon Sting which deals him more damage and, thanks to its angle, can hit him better when he's mid-air. Upon defeating him, X will gain the Storm Tornado\ref{Storm_tornado} and, furthermore, the Death Rogumer will crash land onto the Power Plant Stage cutting the electricity in some section, making the level easier to navigate (moreover the the airship will also disappear from the map screen).
 
 Storm Eagle is 250 cm tall and 135 Kg heavy, while his title is ``\textit{Nobleman of the Skies}''\cite{book:MMX_Complete_art}.
 
 \begin{figure}[h]
 	\centering
 	\includegraphics[width=0.5\linewidth]{figures/X1/Storm_eagle_specs.png}
 	\caption{Storm Eagle specifications according to X1.}
 \end{figure}
\section{Tower}
Among all other stages, the \textit{Tower Stage} (or \textit{Fortress Tower}) is the most different due the fact of extending mostly in vertical. This translate in the player having to climb it via wall-jumping or using moving platforms, with the risk of being hit and fall down at the beginning.

The first part of the stage sees X climbing a series of platform placed in a zig-zag patter and with enemies on them, which will try to attack X while jumping from one to another and make him fall down. Next is a horizontal section in which \hyperlink{enem:Sine_Faller}{Sine Faller} will keep spawning and chasing after X which will have also to deal with laser traps, which will shoot X if he passes trough triggers while active. The traps cannot be destroyed, but it is possible to trigger them and avoid lasers if the player move fast enough. 

The next part is pretty straightforward: it is another vertical section divided into different levels, one on top of another, each one with a \hyperlink{enem:Mega_Tortoise}{Mega Tortoise} in the center which will shoot bombs at X's position. These enemies cover are placed on top of platforms and, due their size, make difficult to proceed undamaged without disposing of them first (although it is still possible by precise dash-jumps). Once on top the player will reach an elevator with spikes all along the walls. As X stands on it, it will start ascending. While rising the elevator will go towards some platform with spike beneath them, which will insta-kill X if he isn't fast enough to dodge them. Furthermore other enemies will spawn and aim at X to obstacle his movements. As the elevator goes up it will start increasing its speed making more difficult to avoid obstacles. Near the end of its ride the elevator will slow down to let X dismount in time, before crashing against the spiked roof. During this part a trick can be performed: if X stands on the rightmost side of the elevator he will avoid all spiked platform, eliminating the major risk factor of this part. Next there are two other climbing sections, one on the outside with moving platform and enemies on top of them, and the last one, inside, again with enemies on top of moving platform and on walls as well. At the end of this last part the player will find the boss door.

This stage contains following enemies\cite{wiki:Tower}:

\begin{itemize}
	\item \hyperlink{enem:Dodge_Blaster}{Dodge Blaster}
	\item \hyperlink{enem:Hoganmer}{Hoganmer}
	\item \hyperlink{enem:Jamminger}{Jamminger}
	\item \hyperlink{enem:Ladder_Yadder}{Ladder Yadder}
	\item \hyperlink{enem:Mega_Tortoise}{Mega Tortoise}
	\item \hyperlink{enem:Sine_Faller}{Sine Faller}
	\item \hyperlink{enem:Slide_Cannon}{Slide Cannon}
	\item \hyperlink{enem:Turn_Cannon}{Turn Cannon}
\end{itemize}

\subsection{Heart Tank}
This stage's Heart tank is hidden in plain sight, right at the end of the outside section,on a large platform near the entrance for the last stage's portion. While being easy to spot, is isn't as easy to get, especially during the stage's first run. Three methods exists to get it. The first and easiest way is to replay the stage after having acquired the Boomerang Cutter and use a boomerang to grub it from the tower's inside, since boomerangs pass trough wall. The second way is to use a charged Shotgun Ice from the entrance and ride the platform while midair, to perform a dash jump from it which will give X enough high to grub to the platform's edge. Finally is is also possible to reach the platform via a pixel-perfect dash-jump which gives X barely enough high to trigger a wall-jump of the platform's edge and subsequently reach its top and the Heart Tank (see \ref{X1:misc} for information on involved tricks).

\subsection{Boomer Kuwanger}\label{boss:Boomerang_Kuwanger}
Boomer Kuwanger (lately renamed \emph{Boomerang} Kuwanger) was a former Maverick Haunter of the 17th Elite Unit under the direct command of Sigma (and also partner of X and Zero in the \mhx remake). However due to his cold, analytic, almost nihilist\cite{book:MH_field_guide}, and cynical vision of the world, he has never had a true sense of justice nor ideals, acting only following logic and his own interest\cite{MHX:manual}. This has lead him to join Sigma's rebellion without interest in its meanings, but the true reasons of his choice has never been explained. Some sources suggest Kuwanger went Maverick and followed Sigma only has it was the most logical action\cite{MHX:manual}, due to his own interest\cite{wiki:Boomer_kuwanger} or even both\cite{book:MH_field_guide}. Other sources even suggest he went Maverick out of fun\cite{Xcoll1:Manual_X1} or out of spite for humans\cite{wayback:X_resources}. Whatever Boomer Kuwanger's reasons were, after joining Sigma he conquered the tower symbol of the city, to convert it into his own base.

In battle Boomer Kuwanger attack with his signature ability, the instant transmission, which allow him to teleport across the arena, typically behind his opponent, to strike him with his horns. This teleportation ability worth him the title of  ``\textit{Blade Demon of Space and Time}''\cite{book:MMX_Complete_art}. Once Kuwanger reappears he will perform one attack between Boomerang Cutter and Death Lift. With the first one he will throw his horns toward X with a curved trajectory which will then travel back to him, as the name suggest, while when performing the second one, Death Lift, he will grub X with his mandibles and throw him into the ceiling dealing large damage. Finally Boomer Kuwanger also has a dash attack, where he dash toward X to damage him.

Dealing with Boomer Kuwanger cannot be easy without his weakness, as the continuous teleporting makes him hard to hit with buster shots. Furthermore the player has to keep moving in the arena to avoid Kuwanger to teleport near him and use his Death Lift attack for massive damage. Moreover once hit Kuwanger will also teleport immediately. This means the player cannot hit him more than once, but it also means that with good aim and timing it is possible to prevent him to do any kind of attack. 

The boss fight takes a turn when fought with Boomer Kuwanger main weakness: the Homing Torpedo\ref{Homing_torpedo}. Since missiles shot with this weapons lock onto enemies they can chase Kuwanger even when teleporting, meaning they will always hit him no matter his position. This drastically reduce the fight complexity as X can literally hide in a corner while shooting torpedoes which will automatically hit Boomer Kuwanger, also with increased damage as they're his weakness, triggering his teleportation and preventing him from doing any kind of action.

According to his specifications Boomer Kuwanger is 242 cm tall and 94 Kg heavy (again, different from in-game information which makes him shorter and lighter). More interesting however is the fact Boomer Kuwanger is one of few reploids known to have family relationship with another reploid (in this specific case a brother): Gravity Beetle.%\ref{Gravity_beetle}.

\section{Power Plant}
\section{Forest}
\section{Sigma Stage (1-4)}
\section{Miscellaneous}\label{X1:misc} %wall clip glithc, seven pixel, auto edge grub flying hadoken %subt-tank + life farming

\part{Characters}
	\chapter*{Preface}
	In this appendix more detailed information about characters who appear in the Mega Man X series are given. Here, however, information will not only comes from in-games material but also from external (but still official) sources which expand and complete what said inside games. Another difference between this part of the document respect previous ones is the fact that while in precedent chapters only fact were reported and, leaving out any interpretation, here the objective is try to put together all information available, even giving a plausible solution in case of conflicting elements. In any case whenever a solution is proposed it will be clarified, to let readers know what parts come from official sources and what instead are created solely for this document.

	\warningbox{\label{assumptions}\small{
	As the document proceed some time reference will be given for some of the events that occurs. However in some occasions the same events can refers to different points in time depending on game's version or the source of the information. While these discrepancies will still be appointed, since the focus here is to give a coherent timeline of events there will be now listed all time-related assumption made in this part, with a short explanation:
	\begin{itemize}
		%\item X's slumber last 100 years (it is never clarified the real duration, but 100 years is always given as minimum value)
		%\item X wakes up at least in 2114, as it is possible to see in the \x opening observing Dr.~Cain's  PC copyright years.
		%\item By the previous two statements, X had to be sealed roughly around 2014.
		\item Maverick Haunters operate for about two years before the first games' event, according to Japanese version of Dr.~Cain's journal, contrary to the two-month stated in the English version of the same source. This due the fact that two month aren't sufficient to explain all event that occur before the first game, such as Zero's awakening, X joining the Haunters or X befriending Zero.
		\item According to the original story, X doesn't join the Haunters until Sigma's revolution begins. Here we deliberately choose the \mhx storytelling, as it allow us to explain better some relationship between characters.
		\item After his death, Dr.~Light consciousness keeps living via AI, which communicates with X via his capsules. This due the fact that in multiple occasions Dr.~Light is fully aware of the circumstances and even directly answer posed questions.
		
	\end{itemize}
	}
	}
	%Mega Man X
\chapter{X}\label{char:X}
X is the main protagonist of the Mega Man series which bares his names. He is the last creation of the brilliant Dr.~Light, a robot built with the gift of free will and limitless potential, and a Maverick Haunter appointed of stopping reploids gone maverick to hurt human people. 
X's primary characteristic is his kind heart and peaceful attitude, which makes him repel violence, often trying to reason with his enemies. However despite this attitude X know well the threat mavericks are and, even if unwilling, fight against them with all his strength to bring the world at peace.

\section{Technical Specifications}
X's specification can be found both in the opening scene of the first Mega Man X game and in the \emph{Rockman \& Rockman X Daizukan}~\cite{book:RRXD}-\cite{X_specs_translated} book. While some information overlap between these sources other are exclusive to a single one, hence here both of them are reported as source of information.

\paragraph{General Information}
\begin{itemize}
	\item Height: 160 cm.
	\item Weight: 57 Kg (lighter than Rock due technological improvements).
	\item Main energy source is Solar energy.
	\item Armor is composed by a lightweight ``Titanium-X'' alloy, the strongest metal in the world. Very light and resistant to heat and shots.
	\item Internal skeleton is a super elastic armor that can reduce received damage by approximately 93\%.
	\item A.I. age between 14 and 15 years old in human terms at the beginning of the first game. Matures as time passes.
\end{itemize}

\paragraph{Head equipment}
\begin{itemize}
	\item Eyes are constituted by broad-range cameras, giving him the ability to see more things human eye can.
	\item Ears are composed by ulta-high sound recognition system, allowing to hear even ultrasonic sounds.
	\item Voice is produced by a voice generation system made by HAYATOM Inc. (MOKUOO Inc. in Japanese version). 
\end{itemize}

\paragraph{Chest equipment}
\begin{itemize}
	\item An Accumulative Energy Generation Device allow to accumulate solar energy and provide X with the sufficient amount of power required to work.
	\item A Micro-fusion fuel tank, which stores fuel for X to use when solar energy is not available, such as caves or underwater.
	\item The Central Joint-controlling system, X's secondary brain which control his movements.
\end{itemize}

\paragraph{Leg equipment}
\begin{itemize}
	\item Gyroscopic Stabilization System/Full auto-balancer helps X in remaining stabilized and land properly from any state he's in.
	\item The Emergency Acceleration System, which enables X to accelerate at high speed in a short amount of time. This equipment is optional and must be installed in a second moment.
\end{itemize}

\paragraph{Arm equipment}
\begin{itemize}
	\item Mega Buster Mk.17 (X-Buster). X’s basic weapon built into his hand. When fighting his hand retracts and leave the place to the buster~\cite{elysium_weapons}
	\item Energy Amplifier to store and concentrate energy into a more powerful shot, the Charged Shot.
	\item Variable Weapon system: Allow the X-buster to transform and emulate attack from enemies bosses. How the copy/learning process is performed remains, however, unknown.
\end{itemize}


\section{Creation}
X's creation begins in the year 20XX by the hand of famous scientist Dr.~Thomas Light. Reasons that brought Dr.~Light are to be found into two main facts that happened during the scientist's life. The first one was the coming of an unknown computer virus from space that caused robots to go violent (which could be a reference both to the ``Evil energy'' appeared in \emph{Mega Man 8} or the \textit{Roboenza} Virus appeared in \emph{Mega Man 10}). After this event Dr.~Light decides that a new battle robot had to be created in order to protect the future of earth~\cite{mega_man_network:Zero_timeline}, and starts building X. On why creating X and not upgrade Rock, the original \textit{Mega Man}, many option can be formulated, one being the fact that Rock was originally designed to help with laboratory work and not to fight, hence the preference to create a new robot, instead altering Rock too much. In addition since the new robot to be created may fight robot infected with viruses, a new anti-virus system had to be created within him. This last statement fits perfectly with the second reason Dr.~Light began creating X, which is his dream to create a robot who could choose his own path in life, effectively having free will. How it is possible to see from his journal~\cite{Dr.Light_journal}, in fact, the idea of a free-will was born and stayed in him from shortly after events of the first \textit{Mega Man} game up until X's creation, as he believed it was his duty to accomplish such achievement\footnote{\textit{If a robot posses the intelligence to be conscious of the possibility of opposing a human for the right reasons, will it be possible for them to worry aver what path is right? [...] I can sincerely feel that coming to think of this is my duty.} Dr.~Light's journal, 29 March 2017}. Furthermore as Roll herself pointed to Light, free will also create the perfect anti-virus system, as would made impossible for a robot to be manipulated\footnote{\textit{``If artificial intelligence can be aware of its own intelligence, then one can fix any kind of tampering'' [...] However if that very kind of electronic intelligence can realize ideas like hers} (Roll) \textit{, then I think it's possible to establish a consciousness that cannot be manipulated} Dr.~Light's journal, 7 March 2017}.  It is unknown how long it took for Dr.~Light to complete X but, observing from flashbacks shown in the \emph{Day of $\Sigma$} OVA, it is possible to imagine that the development took most of Light's remaining lifetime as flashbacks shown a doctor growing older and weaker as time passes.  During this time period however Light realizes the importance of his project and its impact it could have on the world, due the power X holds, making him also realize the danger X could represent if, should the world turn against him, he takes a wrong path in life or begin to question the firs law of robotics risking disasters even worse than ones created in Wily's incidents~\cite{elysium_light_warning}. To avoid this situation, while still believing in X's good heart (as seen in X's flashback in \emph{Day of $\Sigma$}), Dr.~Light decides that X's moral integrity has to be tested deeply before letting him free. According to his studies, Dr.~Light estimates that about thirty years of testing should be necessaries to completely ensure X's safety, time far beyond his own lifespan. For this reason, being near the end of his life and not having anyone who could continue his work, Dr.~Light decides creates a special capsule for X to rest in, capable of performing test without the need of someone supervising it while also ensuring X's safety. After giving X farewell, on date 19$^{th}$ September 20XX, Dr.~Light proceeds to seal him away and leaves a message (written or recorded depending on the game) for whoever will find the capsule, explaining who X is and why he's special.

\section{Awakening and birth of reploids}
X's sleep last for about hundred years inside Dr.~Light's laboratory, buried underground and hidden from everything, until Dr.~Cain, a scientist of 21XX, while searching for plant fossils  from Mesozoic (or preserved plant from Middle Age~\cite{elysium_Cain_journal} in the Japnese version) randomly finds the laboratory. After some digging he manages to get in, where he first finds Light's note and documentation relative to X and, the following day, X's capsule itself still working. After reading Light's last note and warning message, and checking the capsule status, which as stated in the journal show all indicator on green, on the 14$^{th}$ April 21XX Dr.~Cain awakes X from his slumber. Immediately after meeting him, Cain realizes how incredible X is and how futuristic Light's creation  is, even to his times, and decides to bring X, along all Light's design note, to his laboratory in order to try replicate X's design and create a similar robot.

More than six month were needed to complete the first robot, but on the 22$^{nd}$ November Dr.~Cain manages to create a robot using Light's schematics and X as reference. This new robot, just like X, is fully capable of making decision on his own, even arriving to argue with Cain himself, to his surprise. However the new robot created isn't a perfect copy, as part of X's design couldn't be analyze even with modern technologies forcing Dr.~Cain to fix missing elements at his best, especially components constituting his ``\emph{Distress Circuit}''\cite{book:RMZ_Complete_works} which allows X to choose his side in society, and the new robot's moral integrity of the wasn't test deeply as X's one. However since these difference seems to cause no problems, Dr.~Cain decided to start mass production of these new kind of revolutionary robot, which he named ``\emph{Reploids}''  (or \emph{Repliroids}).

It took not so long for reploids to integrate inside the society as Dr.~Cain himself note in his diary in an entry dated 3$^{rd}$ May, where he states that \textit{everyone seems to be happy to accept them}. Reploids in fact began to work in place of and together with humans in all kind of jobs, especially more dangerous ones which could put humans life at risk. However the situation won't last long,  as first mavericks will shortly start to appears.

\section{Mavericks and Maverick Haunters}

In the following entry in Cain's journal dated 16 July the scientist states: ``\textit{Three reploids went "maverick" today and injured two people before they were stopped. This is the third instance of this type of behavior and I still have no idea of what is causing it!}''. This entry describe first mavericks occurrence in the series although, at the time the entry is written, they look more like isolated events related to some sort of explainable fault. However due the problem mavericks had caused, even with such a small number of occurrences (only three), the journal describes how the idea of halting the production already began to spread, but considering how society now depends on reploids, this idea is discarded from becoming reality. Instead a special organization called ``Haunters'' was established in order to track down and halt mavericks before they cause any damage. Appointed leader of the organization is Sigma, Cain's latest and finest work equipped with last-design circuit which should prevent him from any fault, which also serves as leader of the 17th elite unit, operating on front lines against the maverick threat. X will join the organization only later in time (see \PtIIWarning), where will be assigned to the 17th unit under Sigma and where he will meet his future partner, Zero. 

Thanks to haunters' effort any further injure occurred as consequence of maverick attack, creating a situation of peace that lasted  for almost two years (two month in the English, see \PtIIWarning) During this period X fights together with his commander Sigma, his partner Zero and all his companions against mavericks, but deep inside he keeps questioning about his place in life and the path he has to choose\footnote{``\textit{I am a little worried about X. He seems unsure of his place in life and what Dr. Light had planned for him}'' Journal of Dr.~Cain, 10$^{th}$ December 21XX}.

\section{The X saga and the Maverick Wars}
On 4$^{th}$ July 21XX, Dr.~Cain worst nightmare becomes true. On this day Sigma go maverick and begins his revolution against humanity, which he now considers an obstacle to Reploids evolution that has to be eliminated. Together with him sides most of his subordinates Maverick Haunters longing to follow their leader as well as other reploids seduced by Sigma's charisma and strength. In truth Sigma went maverick long before the day he announced his revolution, as he managed to accurately plan his actions and gathered allies for long time. The only difference between Sigma and any other maverick before him is the fact that while any other reploid, as it goes maverick, goes rampage immediately attacking everyone around him, Sigma instead has had a slow conversion which twisted his sight of the world. Further details on Sigma's conversion will be given in future section of this document, while for the moment it is important to underline Sigma's difference with any other maverick that has appeared until this point in time. After rebellion's beginning, the few Haunters remained loyal to their original purpose, X and Zero Included, begin their counterattack. starting a period of wars which in future will be labeled as ``\emph{Maverick wars}''. 

From now on X's story follows the game's plot. During the event of the first game (and its remake) Zero is appointed as leader of the Maverick Haunters, as being the highest in rank still on the right side, and lead other haunter, X included, in fighting. Zero then asks  X to take care of Sigma's subordinate both because their action are causing trouble to humans and other reploids, but most importantly to allow X to grow up and become stronger,as Sigma is still to powerful for him. Once X disposes of the eight mavericks, Zero contact him to ask help to infiltrate Sigma's fortress, which he managed to locate in the meanwhile. Once inside the two first challenge Vile, which result in his death but in Zero's death too. Finally X face Sigma and defeat him, halting his plans of human extinction. After escaping Sigma's collapsing fortress, X return to the Maverick Haunter headquarters, where he becomes the new leader in charge.

	%Zero.tex
\chapter{Zero}\label{char:Zero}
Zero is the deuteragonist of the X series. A close friend of X, Zero acts as a counterpart for him, opposing X's kindness and uncertainty with a cold and emotionless attitude, often taking action without hesitating. However, behind his attitude Zero hides a kind but wounded soul~\cite{wiki:Zero} which cares for his friends.

\begin{figure}[h]
	\centering
	\includegraphics[width=0.4\linewidth]{figures/Characters/Char_Z.png}
\end{figure}

\section{Creation}
Just as X was the last and best creation of Dr.~Light, Zero is the last creation of Dr.~Wily, fulfilling the doctor's dream of creating the strongest robot ever. 
Just like for X, precise information about Zero's creation or specifications are not given. The first instance of Wily talking about Zero's project can be found in Bass' ending in \textit{Mega Man: The Power Battle}, where Wily discusses with Bass about his new project, a robot capable of easily disposing of Bass himself as well as the original Mega Man\footnote{\textit{The robot I'm making right now will blow the both of you away}- Dr.~Wily, Bass ending, Mega Man: The Power Battle}. Then, in the following entry in the series, \textit{ Mega Man 2: The Power Fighters}, this discussion is resumed again, with Wily talking about how he managed to learn from his mistakes, which led him to developing a new type of robot, by combining together the accidental discovery of the bassium energy as a power source or the study of both Proto Man~\cite{art:mmnetwork:24_mavs} and Mega Man:
\begin{quote}
	"I studied Megaman hoping to create a similar robot. Then I developed a powerful energy called "Bassnium" purely by accident. Thus, I created you Bass. Currently Bassnium is the most powerful energy on Earth. But, that's not for long. Hee hee, I've learned from my accident... And I've created a new type of robot which is much more powerful than you or Megaman! It'll be some time before I complete this project though. You better get ready!" - Dr. Wily, Bass ending, Mega Man 2: The Power Fighters
\end{quote}
On such occasion, Wily also shows the blueprint of his project, a shadowy silhouette of Zero with his appearance from X2 onward.

...

\section{The X saga and the Maverick Wars}
As Sigma go Maverick, Zero is appointed as leader of the Maverick Hunters, being the highest in rank remained within the group. He then proceeds to lead the organization against Sigma, entrusting X to deal with Sigma's subordinates while he tries to locate the enemy fortress. The two manage to complete their task simultaneously, and reunite to attack Sigma together, Zero acting as decoy due him being more powerful than X, to let him sneak inside unnoticed. Once in they briefly reunite but are interrupted by Vile which Zero  challenges to a duel, only to be caught in a trap Vile had previously prepared. Vile then proceeds to use him as a hostage and manages to capture X too, but Zero breaks free and, in order to save his friend, explodes to take down Vile too, but only manages to destroy his Ride armor. His action, however, gives X the strength necessary to break free and take Vile down definitively. With his last words Zero encourages X to proceed and face Sigma, firmly believing he has the power to beat him.

Although destroyed, Zero's control circuit remains miraculously intact and is stored at the Maverick Hunters Headquarter. Zero's body (or what remains of it) is instead recovered by the X-Hunters, which tasks Serges to repair it. Serges not only repairs Zero's body, but also upgrades it into its final version, adding the missing shoulder pads and giving him his iconic Z-Saber. For the whole duration of the X-Hunters operation Zero remains incomplete, as the two key components stay in the hands of the two opposing factions. It is only near the end that Zero is resurrected, depending on the action X takes while fighting the X-Hunters. Zero can either be saved by X, who had previously won Zero's body parts from the X-Hunters, and be repaired by Dr.~Cain, or can be resurrected by Serges, after the X-Hunters steal the control circuit from the Maverick Hunters HQ. In the former case he reunites with X shortly before the final fight, just in time to destroy his clone, while in the latter case he will be put under Sigma's control and will fight X, which has to defeat him and make him come to his senses. Whichever the case is, Zero aids X during the final confrontation with Sigma, by destroying the main computer Sigma is using to spread his virus around the world.
	
	%Sigma + Vile.tex
\chapter{Sigma}\label{char:Sigma}
\begin{figure}[htp]
	\centering
	\includegraphics[height=\BIGportraitsize]{figures/X1/Sigma_stages/X_DiVE_Sigma.png}
	\caption{Sigma}
\end{figure}
Sigma is Dr.~Cain's greatest creation. Developed to be a strong leader to face the Maverick threat, Sigma is equipped with the latest design in terms of brain circuit, which according to Cain himself should be fault-safe and prevent him from going maverick. Sadly things don't go as Cain wished, as after solely two years of leading the Maverick Hunter Sigma not only goes maverick, but began a war against humanity bringing on his side a major part of Maverick Haunter which he led. To Sigma, in fact, humans are only an obstacle to reploids evolution, and should be eliminated to allow reploids to reach their full potential. 

In this chapter Sigma' action through the X series will be described. Since in the first game Sigma acts only as the final boss, with very few interactions with X beside his boss quotes, for the first part of his story the description will stay closer to his \mhx version, as it gives more insight on Sigma's personality and motivations.

\section{Leader of the Maverick Hunters}
WIP. Information will be added at the proper time.

\section{The X saga}
Sigma declares his war to humanity on July 4$^{th}$ 21XX, and launches an all-out attack onto Abel city, by deploying his troops to conquer strategical positions and assigning powerful reploids to protect them, but only after having first stroke the city  with a missile attack, as shown in \textit{Day of $\Sigma$}. On this occasion he has a first confrontation with X which Sigma easily defeats. However during the fight for a brief moment X manages to reach his full potential hitting Sigma and leaving his signature scars. After that he retires
into his fortress, waiting for his scheme to be completed. In truth, Sigma's plans not only aim at human extinction, but he also wants to discover X's true potential by forcing him to fight. To no surprise, in fact, X manages to defeat all his subordinates and, with the help of Zero, to infiltrate his fortress. Here Sigma prepare a last test for X, by resurrecting all defeated reploids and making them fight X once more\footnote{``\textit{Sigma must have brought his body back to life}''-X talking to a resurrected Launch Octopus-\cite{wiki:MM_MHX_script}}. X passes this trial too, even taking down Vile, and finally reaches Sigma which happily verifies that he was right about X and reploids' unlimited potential, and then proceeds to challenge him in a final fight, firmly believing to be superior. Sigma however misjudges X, which strikes him down even after he combines with a giant wolf-type mechaniloid (\ref{boss:wolf_sigma}) to further increase his power. Defeated in the body, but not in the spirit, Sigma's body sinks into the sea alongside his fortress while X, teleported outside, watches silently believing to have put an end to the war. Sadly, this couldn't be far from true, as in reality Sigma's true consciousness had already left his original body before the fight with X, preparing to resurrect again in case of defeat: ``\textit{What you defeated was not my true self. The machine that was destroyed was more like another body. I will materialize and resurrect once more}''~\cite{wordpress:X_japanese_script}.

Sigma returns again six months later, at the end of X2. During his absence the command of his forces were entrusted to the X-Hunters, which had their main goal in resurrecting Sigma again, building for him a new body. This is achieved thanks to Serges, the leader of the X-Hunters and a brilliant scientist. Alongside his resurrection, the three mavericks also work on other two schemes: increasing their army size, by the maverick virus throughout the word from the Central Computer and by building new mavericks, and resurrecting the deceased Zero as a maverick. It is not clear however if these two last objectives were posed by Sigma himself or were perpetrated by the X-Hunters autonomously. What is indeed sure is that at some point during this second revolution, Sigma must have received vital information about Zero's origins, since from this point onward alongside his obsession with X he will also develop an obsession for having Zero side with him. Sigma's plans however do not go as expected, since not only Zero does not side with him (depending on the choices made he will either never side with him or snap out of his control after a fight with X) but X  to defeat him again, even after Sigma managed to manifest his true form by using the Central Computer energy to manifest. Defeated again Sigma disappears, leaving behind words of warning for X, as he has resurrected once and will do it again until his victory. However in his last words Sigma can not hold for himself the regret for not understanding why Zero did not joined him: ``\textit{But, Zero, why…he’s…the last of…Wi…num…ers…}''~\cite{wordpress:X2_japanese_script}.
\begin{figure}[htp]
	\centering
	\includegraphics[height=\portraitsize]{figures/X2/Hunter_stages/Sigma_Virus.png}
	\caption{Sigma in his viral form}
\end{figure}

Following his defeat by X, Sigma enters a deep slumber~\cite{wayback:X3_resources}, despite his virus kept spreading and turning reploids in mavericks. Such activity however did not pass unnoticed to scientists, who begin searching the cause of the maverick behavior. Among this scientists, it is Dr.~Doppler who manages to give a proper answer, by isolating what he names the Sigma Virus and developing a proper countermeasure. During such work, however, Doppler becomes infected too by the virus, turning him into a loyal serve of Sigma. Now with a brilliant scientist by his side, Sigma seize this opportunity to rise once again, by forcing Doppler to cause a new rebellion using reploid he was supposed to have cured. Moreover, Sigma also forces Doppler to collect data from the most advanced reploid and use them to build him a new powerful body. Despite everything, however, X still manages to halt Doppler's plans and face Sigma, winning against him once more. Having lost again his body to X, Sigma tries one final attempt to defeat his enemy, by entering his viral form and trying to possess X. He almost succeeds in it, cornering X after a chase, but he is ultimately defeated again by Zero/Doppler, who uses an antivirus developed by the scientist to disperse him once more.


\chapter{Vile}\label{char:Vile}
\begin{figure}[htp]
	\centering
	\includegraphics[height=\BIGportraitsize]{figures/X1/Sigma_stages/X_DiVE_Vile.png}	
	\caption{Vile}
\end{figure}

Although Vile does not play a major role in the X saga except for the first game, his status as recurring enemy in some games plus the presence of a dedicated game mode in the first game's remake, which better describe Vile's personality, are sufficient in order to dedicate this small chapter to him alone. 

Before beginning, however, it is important to appoint how differences between the original Mega Man X games and its remake Maverick Hunter X will be handled. In those games, in fact, Vile's personality is slightly different, since in the former not much information is given beside his background and his dialogues as a boss, while in the latter, also thanks to his dedicated game mode, more details about his personality are given. In this document we will stick to the latter, as it incorporates almost every aspect of the original story too. Clearly it is debatable whether what shown in the Vile's mode is effectively true or not, as the game mode itself is presented as a what-if scenario and hence not canonical. However since this is the only source of additional information about Vile's personality, and for most part of the mode Vile's personality matches with the real one shown in the actual game, what shown about Vile's personality in his mode will be used in the document and considered to be true.

Clearly what said until now regards only the first game of the series, as there aren't any remake for the other games Vile appears in.

\section{Before the war}
It is unknown when Vile was created or who build him. What is sure, however, is that Vile was designed purely for combat, as Zero points to X during their first meeting with him\footnote{``\textit{X, you shouldn't expect to defeat him, he is designed to be a war machine.}''- Zero}, maybe specifically to work as a Maverick Haunter. Thanks to his equipment, Vile quickly became one of the highest-ranked Maverick Haunter of the 17$^{th}$ elite unit (the same as X, Zero and Sigma), with the Special-A (SA) rank, the same as Zero and Sigma, despite his bad attitude. According to information given, in fact, Vile have always behave with arrogance and superiority, only taking orders from himself\footnote{``\textit{I'll tell you one thing$\dots$ I don't like working for others.}''-Vile~\cite{MHX:Vile_script}}, disrespecting superiors\footnote{``\textit{You don't respect authority. You don't follow orders. I pity you}''-Armored Armadillo~\cite{MHX:Vile_script}} and working alone, not having anyone to consider friends and, on the contrary, disliking some of his comrades\footnote{``\textit{I've always hated you, Storm Eagle. You and that smug face of yours}''-Vile~\cite{MHX:Vile_script}}, especially X. Although this he still has the respect of his commilitones, such as Sting Chameleon (``\textit{It's Vile$\dots$ I used to have nothing but respect for you, you know.}''~\cite{MHX:Vile_script}).

It is with X however that Vile shows his worst. To him, in fact, X is only a weak reploid and nothing more, and thus he doesn't understand the reasons why people around him claim X to have incredible power. Because of this Vile develops a grudge and hatred for X, which in reality only covers his jealousy, and dedicates himself to the task of defeating and humiliating him for as much as he can, in order to prove himself and to others that he is the strongest reploids.

Vile's situation gets worse when a fault in his electronic brain occurs. Due to this fault Vile starts to enjoy much more the pleasure of haunting and destroying his target, almost to the point of being an obsession, making him ignore any collateral damage he could cause with his actions and aggravating his position inside the Haunter organization up to the point of being considered to be a borderline maverick. This leads the high command  to preemptive arrest him, in waiting for a sentence on his destiny.

\section{Mega Man X}
Shortly before the beginning of the first game Vile is set free by a soon-to-be maverick Sigma, which asks for Vile help to defeat X as he fear X could interfere in his plans of changing the world\footnote{``\textit{I need your help, to defeat X [$\dots$] in order to ensure our future and speed along our evolution}''-Sigma~\cite{MHX:Vile_script}}. Despite Vile's hate for taking orders, the idea of defeating X is sufficient for him to follow Sigma and help him with his plan. In reality, however, Vile's claimed objective is to follow Sigma only until X's defeat and then to turn against Sigma himself, defeating him to change the world as he desires, as he says to X after battling with him on the highway (``\textit{There's nothing you can do! I'll defeat you and Sigma! Then I'll change the world!}''~\cite{wiki:MM_MHX_script}). In this situation, however, Zero comes and saves X, damaging Vile's armor and forcing him to retreat (showing also that, despite his obsession, Vile still retains a tactical mind). 

It is unknown what Vile does during most parts of the X game, as he only appears later, at the beginning of Sigma fortress. An hypothesis that could be  formulated is that in the meanwhile Vile had worked and upgraded his ride-armor, as during his last encounter he uses a different, customized version of it, and setting up a trap to prevent Zero from stopping him again. This, however, is only a hypothesis without any confirmations.

Vile's final appearance is inside Sigma Fortress, where he attempts again to destroy X, this time capturing Zero first. He almost succeeds in doing it, but he underestimates Zero and X's potential as the first breaks free and destroys his armor, by sacrificing himself, and the latter takes him down. The same fate awaits Vile in his own game mode, but with the addition of a final dialogue between a dying Vile and Sigma. Here Vile's true intentions are finally revealed: he's sole purpose in life has always been defeating X and nothing more, as everything he claimed were only justifications for his actions. In fact, Vile has never had any idea of what to do had he managed to beat X:
\begin{quote}
	SIGMA: What exactly did you plan to do, Vile? Would you stand before me as a Maverick Hunter? Kneel before me and place yourself at my mercy?
	VILE: $\dots$What did I$\dots$ plan to do? Heh$\dots$ thinking about it now, I'm not actually sure
	[$\dots$]
	VILE: I don't care what happens to this world$\dots$ By defeating X\footnote{In Vile's game mode he manages to take down both Zero and X, only to be destroyed because of a surprise attack from Zero, immobilize him and gives X the time to hit him. Hence the reason why Vile claims to have defeated X}, I've validated
	my own existence$\dots$ and that's all that matters to me now.
\end{quote}

\section{Mega Man X3: Vile mk-2}
\begin{figure}[htp]
	\centering
	\includegraphics[height=\portraitsize]{figures/X3/Doppler_stages/X_DiVE_Vile_MK-II.png}
	\caption{Vile mk-2}
\end{figure}
Vile's story did not end in Sigma's fortress, as his remains are, in fact, recovered by unspecified figures (possibly the same people who recovered also Zero's body parts), probably to repair him later in time. It will take more than six months for Vile to come back to life, only after what remained of him ended up in the hands of a corrupted Dr.~Doppler, who proceeds to rebuild and upgrade him into his new form, Vile mk-2. In the process, it is also possible that Vile gets infected with the sigma virus, making him even more unstable than before. This new version of Vile, in fact, is moved only by hatred and vengeance for X and Zero, the causes of his first demise, even to the point of ignoring the orders from his new master Doppler, who required to capture X alive. Vile, instead, act as he wishes, preferring to lure X into a mined factory and face him instead of actively chasing X as the nightmare police does. Ironically, some sources seem to imply Vile's deep hate for X is much closer to human behavior than X's way to be~\cite{Xcoll1:Manual_X3,wayback:X3_resources}, in a sense making Vile better than X. However, Vile is blinded by the hatred and isn't capable to acknowledge such fact. Despite everything, however, as pointed out in~\cite{wayback:X3_resources}, Vile  doesn't seem to have gone completely mad, has he still shows some glimpse of sanity, for example by preferring to lure X into a trap he prepared (the mined factory) instead of charging forward. Moreover, even after being defeated by X the first time, Vile opts to retreat and reorganize, by building his own personal Ride Armor to use against X. In this settings, Vile also proves again his skills in Ride Armors technology, by building yet another custom armor (the fourth one), the ``Goliath'', specifically made for combat and that Vile's refers to as the world's most advanced Ride Armor~\cite{book:MH_field_guide}.  After having completed his plan, Vile begins to wait for X inside Doppler's laboratory, knowing perfectly X will come to stop Doppler. In the wait, however, Vile decides to let loose to his destructive impulses, wrecking havoc inside the laboratory's section he was supposed to protect. 

In the end, Vile and X meet again and have a final clash. Vile employs all his arsenal, Ride Armor included, but is ultimately defeated again by X. Before dying, Vile manages only to unleash a final warning, telling X that no matter how, he will always hunt him until X has finally died\footnote{``\textit{Don’t think that this is the end, X! I will haunt you to the day you die$\dots$.}''-Vile mk-2~\cite{wordpress:X3_japanese_script}}.

	%Other characters.tex
\chapter{Other Characters}
Here other main characters will be described that don't have enough information to dedicate a whole chapter
about, but that still play an important role in the Mega Man X saga.

\section{Dr. Cain} \label{char:Cain}

\begin{figure}[h]
	\centering
	\includegraphics[width=0.5\linewidth]{figures/Characters/Char_Cain_MHX.jpg}
	\caption{Dr.~Cain as he appears in \emph{Day of $\Sigma$}.}
\end{figure}

Dr.~Cain is the brilliant scientist father of reploid's technology. Originally an archaeologist, Dr. Cain accidentally find Dr.~Light's laboratory while searching from prehistoric plant fossils, with X's capsules and blueprints inside~\cite{wiki:Cain_journal}-\cite{elysium_Cain_journal}, whom he befriends with. By using X's help, Dr.~Light's schematics and his knowledge of robotics, Dr. Cain manages to create his own version of a robots with free-will, which he label ``Reploids''. These new type robots greatly impacted on society, almost at the point of becoming necessary, in a very short amount of time and making Dr.~Cain one of the most important person of his world. However although Dr.~Cain's intention were surely good, aiming to achieve the same dream Dr.~Light had to create a society where humans and robots can life together, his actions were also the cause of firsts Maverick attacks~\cite{book:MH_field_guide}. In creating reploids, in fact, Cain didn't manage to fully recreate X's components, especially his ``Distress Circuit''~\cite{book:RMZ_Complete_works} and had to develop substitutes which where however prone to errors. Moreover these reploids' moral integrity wasn't tested for a period of time as long as X's one, making them more susceptible to taking a wrong path and going Maverick. Dr.~Cain seems partially to acknowledge his error, as after first mavericks appear he tries to find the cause of these problems\footnote{``\textit{This is the third instance of this type of behavior and I still have no idea of what is causing it!}'' - Journal of Dr.~Cain, $16^{th}$ July} develop more advanced reploids with more robust circuits to prevent errors, Sigma being the last of this series\footnote{``\textit{Sigma is one of the most intelligent reploids I've created and contains my latest circuit designs. His systems should be immune to any problems}'' - Journal of Dr.~Cain, $20^{th}$ November}. However even his latest designed circuits didn't manage to keep Sigma safe from going Maverick. Once Sigma begins his assault, Cain remains powerless to watch the destruction his creation caused believing nothing could stop him, not even X or Zero. However he doesn't either stop them from trying, as  he firmly believes something had to be done\footnote{``\textit{ I'm doubtful of their chances ( X and Zero), but I won't stop him. Something has to be done}'', Journal of Dr.~Cain, $4^{th}$ July}.

A different fate is instead reserved to Dr.~Cain's in the series remake, Maverick Hunter X. Here Cain's role remains the same up until Sigma's revolution, with the only difference of being much older and weaker, as he's connected to a life-support machine to extend his life for as much as he can. His only appearance is in the \emph{Day of $\Sigma$} OVA, where he explains to a pre-revolution Sigma the power X possesses and how it manifests in his hesitations and empathy, thus making Sigma interested in him. Cain is then seen only at the end of the OVA, after Sigma declares his war by launching missiles onto Abel city. He is last seen in his house, right before a missiles strikes directly onto it presumably killing him, pondering if reploids, created by humans but with abilities far beyond theirs, were fruit of mankind's arrogance rather than their good intentions and wish for knowledge\footnote{``\textit{Reploids... created by humanity, yet possessing abilities far beyond our own$\dots$ [$\dots$] Mankind's arrogance?$\dots$No}''-Dr.~Cain, Day of $\Sigma$, scene 4}.

%FROM HERE X2

\section{Serges} \label{char:Serges}
Despite not playing a significant role in the game he appears, Serges character and his links to other characters in the series have to be appointed and discussed. In particular what is worth to talk about are the possible connections that subsist between him and Dr.~Wily. By reading the original Japanese script for the X2 games and by studying also external sources which talk about Serges it is, in fact, possible to draw a line which connect these two character albeit said connection has never been explicitly appointed. In this small sections the main evidences regarding this theory will be given.

The first possible connection between Serges and Wily comes form the original script for the X2 game. As, in fact, explained at the beginning of chapter~\ref{cha:X2}, the game underwent a massive localization causing an alteration in dialogues and the removal of some links in the finished product. In this context two are the main dialogues to focus on. The first one is the moment in which the X-Hunters contact X to challenge him in a fight. In the original script Serges open his phrase by first calling X by his ``full'' name\footnote{\textit{Serges: … …crrrk… …bzzzt….Rock…E…cks…}- Serges,~\cite{wordpress:X2_japanese_script}} of \textit{Rockman X}, whereas no other character in the game addresses him in this way, Dr.~Cain included. The second, and probably most important, dialogue to be examined is Serge's final speech after being defeated the second time. In such occasion the original script report the following phrase:
\begin{quote}
	SERGES: Am I to perish here? Defeated by Light’s memento robot again… how regretful…~\cite{wordpress:X2_japanese_script}
\end{quote}
From this phrases Serges once again nods to the fact that he knows X and his origins more than any other. He not only, in fact, shows to know that X was built by Dr.~Light, information only Dr.~Cain knows, but also feeling regret for being defeated once more from a robot built by Light.  This feeling of regret, combined with the knowledge about X's past seems to point in the direction of Serges being, in some form, connected with Dr.~Wily, the original antagonist of the classic series.

Beside the original script, there are also other evidences which seems to point in this direction. According to the information given by the Rockman X2 Collected Sourcebook Information~\cite{wayback:X2_resources}, in fact, Serge's surpasses even Sigma's and tie with Dr.~Wily's one. Furthermore from the same source it is possible to know that Serges was also the responsible for the construction of Sigma's new body (a hard task considering the original Sigma was Dr.~Cain's best work) but more importantly for the rebuilding of Zero's body. Serges manages, in fact not only to fully reconstruct Zero's body but also to upgrade it in its final form, such as by adding the shoulder pads, and also providing him with the signature Z-Saber. Beside the restoration, a task even Dr.~Cain refused to do due how complicate Zero's design his\footnote{A truth which will hold up to X6 and beyond, where X and Zero's body will be considered a mistery}, it the upgrade process the one on which pause. Except for Zero's creator, in fact, no one would be able to know how upgrade Zero or how he should look like once completed, especially since no one could know that Zero was unfinished in first place. Since Zero's creator had been confirmed multiple times over various game, some more canon than others, to be Dr.~Wily (more details about this are given in chapter~\ref{char:Zero}), this strengthens even more the connection between Serges and Wily. In particular in the game Mega Man 2: The Power Fighters, in Bass' ending, a blueprint of Zero with his aspect from the X2 games onward appears, labeled as Dr.~Wily's new robot in development.  The theory of Serges and Wily being the same thing would also allow to explain how Sigma, in his final speech, acknowledge Zero's origins\footnote{\textit{But, Zero, why…he’s…the last of…Wi…num…ers…}-Sigma~\cite{wayback:X2_resources}}, which Sigma could not known if not for someone (such as Serges) had previously told him about.

It is however to underline that despite the evidence presented, the relationship between Serges and Wily is and will remain a mystery since, as stated in the Mega Man X Official Complete Works by Inafune himself, ``\textit{this is one of those things that is best left without an official comment}''~\cite{book:MMX_Complete_art}.

\appendix
\chapter {Vehicles}
\begin{itemize}
	\item \hypertarget{vehicle:Death_Rogumer}{\textbf{Death Rogumer:}}
	An Aerial battleship made for the Maverick Hunter's air cavalry. Originally made in order to hold down reploid rebellions, Sigma converted it as a fortress 
	for his army, entrusting the command to the former leader of 7th Airborne unit, Storm Eagle~\cite{wayback:X_resources}.
	\begin{figure}[htp]
		\centering
		\includegraphics[width=\linewidth]{figures/X1/Storm_eagle/DeathRogumer.jpg}
		\caption{Death Rogumer}
	\end{figure}
	\item \hypertarget{vehicle:Ride_Armor}{\textbf{Ride Armor:}}
	Ride Armors are mechas similar to tanks with attached hands and feet. Originally these machines were made intended to be used in engineering~\cite{wayback:X_resources}, but were later used also for fighting, as they greatly increase the power of their user due being able to dash, walk over spikes, deliver powerful attacks and receive damage in place of 
	their pilot. Starting from this base version many others will be created, focused more on combat power. Vile himself uses two modified Ride Armors in his confrontation with X, although both of them were destroyed by Zero.
	\begin{figure}[htp]
		\centering
		\includegraphics[width=0.5\linewidth]{figures/X1/Enemies/ArmorSoldier.jpg}
		\caption{Ride Armor 's artwork (with pilot)}
	\end{figure}
	
	\item \hypertarget{vehicle:Ride_Armor_Rabbit}{\textbf{Ride Armor EG-2 custom "\textit{Rabbit}":}}
	
	
	\item \hypertarget{vehicle:Ride_Chaser_Cheval}{\textbf{Ride Chaser ADU-T400 turbo "\textit{Cheval}": }}	
\end{itemize}



\chapter{Enemies}
%enemies.tex
Here a list of all enemies with a short description is given. For the X1 enemies information are obtainable from \cite{wayback:X_resources}, while artworks comes from \cite{book:MMX_Complete_art}.

\section{Mini-Bosses}
	\begin{itemize}
		\item \hypertarget{miniboss:Anglerge}{\textbf{Anglerge}}:
		Anglerge are angler-type mechaniloids that work to cleaning the seabed floor, with a  a motion sensor attached to its "lantern" part. Anglerge's attack are only four: send four snake-like robots toward X (two upward and two downward), which travel horizontally and descend when above (or below) X; a vacuum attack to push or pull X toward a spike pit; a (rare) energy beam from the angler (which can be disabled if the lamp is broken via an attack).
		\begin{figure}[htp]
			\centering
			\includegraphics[width=0.4\linewidth]{figures/X1/Enemies/Anglerge.jpg}
			\caption{Anglerge's artwork}
		\end{figure}
	
		\item \hypertarget{miniboss:Bee_Blader}{\textbf{Bee Blader}}:
		A large bee-type helicopter which was created in order to carry \hyperlink{enem:Ball_De_Voux}{Ball de Voux}. It is equipped with a vulcan machine-gun and homing missiles. This mechaniloid has been created for guerrilla operations in forests and cities. While they don't appears formidable enemies, they can be rather dangerous, especially if X defeats them while standing below, as they will fall and crush him instantly.
		\begin{figure}[htp]
			\centering
			\includegraphics[width=0.5\linewidth]{figures/X1/Enemies/BeeBlader.jpg}
			\caption{Bee Blader's artwork}
		\end{figure}
	
		\item
		 \hypertarget{miniboss:Cruiziler}{\textbf{Cruiziler}}: Whale mechaniloid who patrols the sea with its powerful weapon. Some kind of mistake caused it to lose its sea navigation, its attack circuits began running wild, and communications were lost. Its body is totally invincible, save for its core on top.
		\begin{figure}[htp]
			\centering
			\includegraphics[width=0.4\linewidth]{figures/X1/Enemies/Cruiziller.jpg}
			\caption{Cruiziller's artwork}
		\end{figure}
		
		\begin{figure}[htp]
		\centering
		\includegraphics[width=0.3\linewidth]{figures/X1/Enemies/MoleBorer.jpg}
		\caption{Mole Borer's artwork}
		\end{figure}
	
		\item \hypertarget{miniboss:Mole_Borer}{\textbf{Mole Borer}}:
		Mechanioid used to open up paths in mines, using a 
		rotary roller to destroy rocks that obstacle his path. Its armoring can take a lot of damage, while the roller is completely invincible and can instantly kill X. The only way to deal with it is to attack from behind, although several shots are needed to take it down. Using the Fire Wave is the best option, as its continuous damage can dispose of it quickly.
	
		\item \hypertarget{miniboss:RT-55J}{\textbf{RT-55J}}: In times of peace, it was a professional robot sumo wrestler and a popular Yokozuna (sumo grand champion) in the "Robot Grand Sumo Tournament". Moved in the forest, it now guards X's  Chest Parts. Its certain kill technique, the "Kagizume Beam Hand," strikes and tosses its opponents but only if it is in his claw's reach range. Otherwise he'll just jump at it to close the gap. 
		\begin{figure}[htp]
			\centering
			\includegraphics[width=0.3\linewidth]{figures/X1/Enemies/RT-55J.jpg}
			\caption{RT-55J's artwork}
		\end{figure}
	
		\item \hypertarget{miniboss:Thunder_Slimer}{\textbf{Thunder Slimer}}: Thunder Slimer was born from a single question: ``How large can a single cell become?'' This monster was born from said experiment. Its body is over three times as large as X, but may require approximately 10 years before it reaches full growth. It has settled in the power plant, where he absorb electricity and use it to perform electric attack against X.
		\begin{figure}[htp]
			\centering
			\includegraphics[width=0.3\linewidth]{figures/X1/Enemies/ThunderSlimer.jpg}
			\caption{Thunder Slimer's artwork}
		\end{figure}
	
		\item \hypertarget{miniboss:Utuboros}{\textbf{Utuboros}}: Serpent-type mechaniloid made to explore the ocean floor. Thanks to its flexible body it can zig-zag into difficult underwater areas, and burrow underground. His body is totally invincible and can work as a platform, while only the head and tail are vulnerable and can damage X.
		\begin{figure}[htp]
			\centering
			\includegraphics[width=0.4\linewidth]{figures/X1/Enemies/Utuboros.jpg}
			\caption{Utuboros's artwork}
		\end{figure}
	\end{itemize}

\section{Minor enemies}
\begin{itemize}
	\item[{\includegraphics{figures/X1/Enemies/sprite_amenhopper.png}}] \hypertarget{enem:Amenhopper}{\textbf{Amenhopper}}:
	Originally designed for farm work, it was used to sow fertilizer 
	across the land. Now, it's been remodeled into a bomb-dropping
	battle type mechaniloid.
	
	\item[ {\includegraphics[height=20px]{figures/X1/Enemies/sprite_armor_soldier.png}}] \hypertarget{enem:Armor_Soldier}{\textbf{Armor Soldier}}: Lowest class of soldier reploids, used in military affairs. Riding in their Ride Armor, they do destruction work under Sigma's orders.
	
	\item[{\includegraphics[width=30px]{figures/X1/Enemies/sprite_axemax.png}}] \hypertarget{enem:Axe_Max}{\textbf{Axe Max}}: Woodcutter reploid from the forest, remodeled for brutality. Swinging his large axe, he attacks by sending the chopped wood flying.
	
	\item[{\includegraphics[height=20px]{figures/X1/Enemies/sprite_balldevoux.png}}] \hypertarget{enem:Ball_De_Voux}{\textbf{Ball De Voux}}: With 2 soft-treading feet, this mecha can move over any topography.
	Inside the sphere there is a camera and a sensor which can even see in the dark.
	
	\item[{\includegraphics{figures/X1/Enemies/sprite_battonbone.png}}] \hypertarget{enem:Batton_Bone}{\textbf{Batton Bone}}: Bat mechaniloids with a taste for humans. They dwell in forests and caves.
	
	\item[{\includegraphics[height=15px]{figures/X1/Enemies/sprite_battonm501.png}}]  \hypertarget{enem:Batton_M-501}{\textbf{Batton M-501}}: Bat type mechaniloid which the \hyperlink{enem:Batton_Bone}{Batton Bone} series is based on. It is a very unusual mechaniloid, made over 30 years ago. 
	
	\item[{\includegraphics{figures/X1/Enemies/sprite_bombeen.png}}]\hypertarget{enem:Bomb_Been}{\textbf{Bomb Been}}: Small bee-modeled helicopter used for land mines scattering. Able to infiltrate any area, it can set up land mines anywhere.
	
	\item[{\includegraphics[height=20px]{figures/X1/Enemies/sprite_cragman.png}}] \hypertarget{enem:Crag_Man} {\textbf{Crag Man}}: Crag Men were made to clear rock debris during landslides. They works actively with the aerial mechaniloid \hyperlink{enem:Sky_Claw}{Sky Claw}.
	
	\item[{\includegraphics[height=10px]{figures/X1/Enemies/sprite_creeper.png}}] \hypertarget{enem:Creeper} {\textbf{Creeper}}: An insect-type mechaniloid. It's unknown what it was made for.	It is pecked out from the insides of trees by the \hyperlink{enem:Mad_Pecker}{Mad Pecker}.
	
	\item[{\includegraphics[height=20px]{figures/X1/Enemies/sprite_crusher.png}}] \hypertarget{enem:Crusher}{\textbf{Crusher}}: Construction mechaniloid used for knocking down buildings. It drops its steel-made weight to scrape down the highway.
	
	\item[{\includegraphics[height=20px]{figures/X1/Enemies/sprite_diglabour.png}}] \hypertarget{enem:Dig_Labour}{\textbf{Dig Labour}}: The greatest pickaxe worker in the world. He is a diligent reploid who works in the robot factory.
	
	\item[{\includegraphics[height=20px]{figures/X1/Enemies/sprite_dodgeblaster.png}}] \hypertarget{enem:Dodge_Blaster}{\textbf{Dodge Blaster}}: Latest model of mobile cannon with "self-defense function", which make possible to avoid energy attacks before they can even get near it.
	
	\item[{\includegraphics[height=20px]{figures/X1/Enemies/sprite_flamer.png}}] \hypertarget{enem:Flamer}{\textbf{Flamer}}: High-temperature blaze-blowing flamethrower machine. A remodeled airport fire extinguisher mechaniloid, turned into a weapon which tries to spread fires.
	
	\item[{\includegraphics[height=30px]{figures/X1/Enemies/sprite_flammingle.png}}] \hypertarget{enem:Flammingle}{\textbf{Flammingle}}: Flamingo-type mechaniloid taken from the robot zoo. It attacks by spinning its head and releasing the saw.
	
	\item[{\includegraphics[height=20px]{figures/X1/Enemies/sprite_gulpfer.png}}] \hypertarget{enem:Gulpfer}{\textbf{Gulpfer}}: Once the ornamental mascot mechaniloid of a seaside Chaya teahouse, it escaped and was converted for catching ocean fish. It was originally based on an old children's toy.
	
	\item[{\includegraphics[height=20px]{figures/X1/Enemies/sprite_gunvolt.png}}] \hypertarget{enem:Gun_Volt}{\textbf{Gun Volt}}: Mechaniloid developed for military use. A tank made for terrestrial combat, it attacks with missiles and high voltage bullets.
	
	\item[{\includegraphics[height=20px]{figures/X1/Enemies/sprite_hoganmer.png}}] \hypertarget{enem:Hoganmer}{\textbf{Hoganmer}}: Fighter in the future grappling show "Robot Coliseum." It blocks the attacks of enemies with its shield, and attacks	by swinging its iron ball and chain.
	
	\item[{\includegraphics[height=20px]{figures/X1/Enemies/sprite_hotarion.png}}] \hypertarget{enem:Hotarion}{\textbf{Hotarion}}: A mechaniloid for nighttime patrol, it was made to save the firefly appearance from extinction. Shining, it flies through the sky.
	
	\item[{\includegraphics[height=20px]{figures/X1/Enemies/sprite_jamminger.png}}] \hypertarget{enem:Jamminger}{\textbf{Jamminger}}: Mechaniloid that attacks any enemies who enter a forbidden area. An odd robot who laughs after attacking.
		
	\item[{\includegraphics[height=20px]{figures/X1/Enemies/sprite_ladderyadder.png}}] \hypertarget{enem:Ladder_Yadder}{\textbf{Ladder Yadder}}: Originally a mechaniloid supervisor of the forest regions. It would locate any poachers, and report the forest's temperature and humidity to the woodland protection center.
	
	\item[{\includegraphics[height=20px]{figures/X1/Enemies/sprite_liftcannon.png}}] \hypertarget{enem:Lift_Cannon}{\textbf{Lift Cannon}}: Rotary-type cannon attached to a a tube-like stand. Originally, a fire-fighting robot for control towers and any other high places in the airport.
	
	\item[{\includegraphics[height=20px]{figures/X1/Enemies/sprite_madpecker.png}}] \hypertarget{enem:Mad_Pecker}{\textbf{Mad Pecker}}: Woodpecker-type repliroid who chops trees in the forest. Tries to follow \hyperlink{enem:Planty_Iworms}{Planty}, without success.
	
	\item[{\includegraphics[width=30px]{figures/X1/Enemies/sprite_megatortoise.png}}] \hypertarget{enem:Mega_Tortoise}{\textbf{Mega Tortoise}}: A turtle-type mechaniloid originally meant for rescuing humans from maritime disasters. From its back, it now produces bombs in place of floating devices.
	
	\item[{\includegraphics{figures/X1/Enemies/sprite_metalwing.png}}] \hypertarget{enem:Metal_Wing}{\textbf{Metal Wing}}: A reconnaissance mechaniloid. When it spots dangers, it raises its flying speed in a great rush to get news to its master.
	
	\item[{\includegraphics{figures/X1/Enemies/sprite_mettalc15.png}}] \hypertarget{enem:Metall_C-15}{\textbf{Metall C-15}}: Reploid who watches factories. From the former series that worked in factories, now they are advanced enough to be placed as chiefs.
	
	\item[{\includegraphics{figures/X1/Enemies/sprite_planty.png}}] \hypertarget{enem:Planty_Iworms} {\textbf{Planty\&Iworms}}: Planty is from the Mettool family and watches over the forest. From its head, it can manufacture the earthworm-type, soil cultivation reploid, Iworm.
	
	\item[{\includegraphics{figures/X1/Enemies/sprite_raybit.png}}] \hypertarget{enem:Ray_Bit}{\textbf{Ray Bit}}: Rabbit-type mechaniloid taken from the robot zoo. It skips and jumps, using the laser cannon in its ears to attack.
	
	\item[{\includegraphics[height=15px]{figures/X1/Enemies/sprite_raytrap.png}}] \hypertarget{enem:Ray_Trap}{\textbf{Ray Trap}}: Mechaniloid devices which await the false steps of intruders.
		
	\item[{\includegraphics[width=30px]{figures/X1/Enemies/sprite_roadattacker.png}}] \hypertarget{enem:Road_Attackers}{\textbf{Road Attackers}}: A destructive reploid gang of hot-rodders, riding for Sigma's rebellion. Large beam cannons have been attached to the bonnets of their sports cars.
	
	\item[{\includegraphics{figures/X1/Enemies/sprite_rollinggabyoall.png}}] \hypertarget{enem:Rolling_Gabyoall}{\textbf{Rolling Gabyoall}}: Intruder repulsion robot. It Appears to be a simple mechaniloid, but truthfully, it possesses the human-like mind of a reploid.
	
	\item[{\includegraphics{figures/X1/Enemies/sprite_rushroader.png}}] \hypertarget{enem:Rush_Roader}{\textbf{Rush Roader}}: Leaders of the robot gang of hot-rodders. To get revenge on the Maverick Hunters who once chased them down, they became Sigma's subordinate.
	
	\item[{\includegraphics[height=20px]{figures/X1/Enemies/sprite_scraprobo.png}}] \hypertarget{enem:Scrap_Robo}{\textbf{Scrap Robo}}: A pathetic upper body of a robot, made to become a car driver. Although it passed part of the humans' expectations, without a driver's license, it has been turned into scrap.
		
	\item[{\includegraphics[height=20px]{figures/X1/Enemies/sprite_seaattacker.png}}] \hypertarget{enem:Sea_Attacker}{\textbf{Sea Attacker}}: Seahorse-type mechaniloid created as a novelty for humans' homes. Its body somersaults as it charges.
	
	\item[{\includegraphics[height=20px]{figures/X1/Enemies/sprite_skyclaw.png}}] \hypertarget{enem:Sky_Claw}{\textbf{Sky Claw}}: A robot who removes obstacles, originally designed for the ``Crane Game'' which was popular in Japan during the later half of the twentieth century.
	
	\item[{\includegraphics{figures/X1/Enemies/sprite_sinefaller.png}}] \hypertarget{enem:Sine_Faller}{\textbf{Sine Faller}}: Aerial mechaniloid made with the idea "Quality from quantity". It flies and turns, acting as a hindrance.
	
	\item[{\includegraphics[width=30px]{figures/X1/Enemies/sprite_slidecannon.png}}] \hypertarget{enem:Slide_Cannon}{\textbf{Slide Cannon}}: Defensive artillery, set up to attack aerial enemies. Designed after the German anti-aircraft cannons of the 1940s.
	
	\item[{\includegraphics[height=20px]{figures/X1/Enemies/sprite_snowshooter.png}}] \hypertarget{enem:Snow_Shooter}{\textbf{Snow Shooter}}: Bad-natured mechaniloid who toss balls of white iron as if they were snowballs. They are Chill Penguin's guardians.
	
	\item[{\includegraphics[height=20px]{figures/X1/Enemies/sprite_spiky.png}}] \hypertarget{enem:Spiky}{\textbf{Spiky}}: Monocycle which bears sharp spikes in its tire. Very dangerous, its main attack technique is to slide over 
	and self destruct.
	
	\item[{\includegraphics{figures/X1/Enemies/sprite_tombot.png}}] \hypertarget{enem:Tombot}{\textbf{Tombot}}:A dragonfly-type glider. When taking off, it cuts and releases the jet propulsion units. Then, slowly riding the wind, it flies through the sky.
	
	\item[{\includegraphics[width=20px]{figures/X1/Enemies/sprite_turncannon.png}}] \hypertarget{enem:Turn_Cannon}{\textbf{Turn Cannon}}: Robot once designed as a sprinkler for domestic use, but was defective until the water was replaced with cannon shells.

\end{itemize}












\chapter{Timeline}
%Timeline.tex
Here a timeline of all events concerning the Mega Man X series in chronological order. Events which happen at a precise date have it clarified on the right.
\warningbox{\label{time_assumptions}\small{
		Since most of the events presented here don't have a precise date, the order they're placed is only according to the writer's preference. Any other decision about event's ordering has been taken according to \PtIIWarning, or will be explicitly addressed.
}}

\begin{tabularx}{\linewidth}{l X}
	
	\toprule
	\rowcolor{Aquamarine}
	%
	% 20XX
	%
	\multicolumn{2}{c}{\textbf{20XX}}\\
	\addlinespace[1.5ex]
	\tabdot&29$^{th}$ March 20XX: Dr.~Thomas Light begins pondering the idea of a robot capable of choosing his own path in life. \\
	\tabdot& Dr.~Light begins to construct X.\\
	\tabdot& Dr.~Wily begins the construction of Zero, using Rock's and Proto Man's schematics. The objective is to create a robot capable of surpassing both Rock and Bass.\\
	\tabdot& Dr.~Wily finishes to build Zero, but the robot is uncontrollable and attacks everything in sight. Wily decides to seal him away and begins working on a way to correct Zero's behavior.\\
	\tabdot& Dr.~Light finishes to build X, but seals him away in a capsule which can run the thirty-years required tests in autonomy. He then leaves a warning message for whoever will find X.\\
	\tabdot& Dr.~Light dies. His consciousness, however, lives on inside and AI which continue to work on the X project developing the First Armor.\\
	\tabdot& Dr.~Wily dies. His consciousness manages however to live on.\\
	\tabdot& The Batton M-501 mechaniloid series is discontinued, in favor of the newer Batton Bone one.\\
	
	%
	% 21XX
	%
	\midrule
	\rowcolor{Aquamarine}
	\multicolumn{2}{c}{\textbf{21XX}}\\
	\addlinespace[1.5ex]
	\tabdot& The Met series is upgraded with the introduction of Metall C-15.\\
	\tabdot&10$^{th}$ April 21XX: Dr.~Cain accidentally finds Dr.~Light's laboratory buried underground while in a archaeological expedition.\\
	\tabdot&13$^{th}$ April 21XX: Dr.~Cain finds X's capsule.\\ 
	\tabdot& 14$^{th} $April 21XX: X awakens.\\
	\tabdot& 15$^{th}$ April 21XX: Dr.~Cain, alongside X and Light's schematics, moves to his laboratory in order to try to create a new type of robot following Dr.~Light's designs.\\
	\tabdot& 22$^{nd}$ November 21XX: Dr.~Cain completes his first reploid.\\
	\tabdot& Cain Industries begin reploids' mass production.\\
	\tabdot& The first instance of a reploid going maverick occurs.\\
	\tabdot& Cain creates Sigma, a strong reploid equipped with last-design brain circuits.\\
	\tabdot& 20$^{th}$ November 21XX: The Council decides to set up a special organization, named ``Maverick Hunters'' with the specific task to hunt down and stop mavericks. Sigma is appointed as the leader.\\
	\tabline& $\cdot$ Sigma is appointed leader of the 17th Elite unit.\\
	\tabline& $\cdot$ Launch Octopus, Wheel Gator and Bubble Crab join the 6th Fleet.\\
	\tabline& $\cdot$ Chill Penguin joins the 13th Polar Division, stationed in the South Pole.\\
	\tabline& $\cdot$ Armored Armadillo is appointed commander of the 8th armored force.\\
	\tabline& $\cdot$ Flame Mammoth becomes captain of the 4th Land Unit stationed in the Middle East.\\
	\tabline& $\cdot$ Storm Eagle becomes leader of the 7th Airborne Unit. Overdrive Ostrich joins the same unit.\\
	\tabline& $\cdot$ Boomer Kuwanger, Gravity Beetle, Flame Stag and Spark Mandrill join the 17th Elite Unit.\\
	\tabline& $\cdot$ Sting Chameleon is assigned to the 9th Special Force (Ranger Unit).\\
	\tabline& $\cdot$ Magna Centipede joins the 0th Special Unit ``Shinobi''.\\
	\tabline& $\cdot$ Vile joins the 17th Elite Unit\footnote{Here the fact that all mavericks joins the organization from the beginning is only author's flavor as no information about these events are given}.\\
	\timepoint{Scientists build Crush Crawfish as a combat reploid. However a bug in his AI makes him incredibly violent. Hence, he is sealed away in a warehouse for later repair.}
	\midrule
	\tabdot& 10$^{th}$ December 21XX: Maverick Hunters have been operating for about two years. X is still seeking his place in life.\\
	\tabdot& Overdrive Ostrich has an accident, which deprives him of his flight ability. Ostrich retires from the Hunters.\\
	\tabdot& Zero awakes and joins the 17th Elite Unit\\
	\tabdot& X joins the 17th Elite Unit.\\
	\tabdot& Wheel Gator attacks a comrade, and is forced to leave the Hunters while being chased.\\
	\tabdot& Sigma begins to actuate his plans.\\
	\tabdot& A fault in Vile's brain circuits makes him unstable and borderline-maverick. The high ranks of the Maverick Hunters decides to jail him waiting for a judgment.\\
	\tabdot& Sigma calls back Chill Penguin from the South Pole, making him temporarily join the 17th Elite unit. Before coming back, Chill Penguin eliminates his former superior.\\
	\tabline& $\cdot$ Flame Mammoth is recalled too, but he doesn't join the 17th Elite Unit.\\
	\tabdot&Sigma reaches for Morph Moth, being interested in his power of evolving through absorbing scraps, and enrolls him into his army. \\
	%
	% day of sigma
	%
	\midrule
	\rowcolor{Aquamarine}
	\multicolumn{2}{c}{\textbf{Day of $\Sigma$ and beginning of the war}}\\
	\addlinespace[1.5ex]
	\tabdot& Vile is arrested, suspected of being a maverick.\\
	\tabdot& Sigma frees Vile, asking him to take care of X.\\
	\tabdot& 4$^{th}$ July 20XX: Sigma attacks a missile base. X and Zero try to stop him, but they fail. Sigma fires missiles onto Abel City, effectively starting the war. During the fight X hit Sigma, causing him scars onto his eyes.\\
	\tabdot& Storm Eagle and his troops attack Sigma, trying to take him down. Sigma wins, putting him under his rule. Storm Eagle is then entrusted with the control of the Death Rogumer.\\
	\tabdot& Sigma's subordinate conquer various strategic points.\\
	\tabline& $\cdot$ Launch Octopus sets up a base in the ocean, to cut down naval transports. Bubble Crab defects too, but remains in the rears.\\
	\tabline& $\cdot$ Chill Penguin occupies a base in a nearby mountain to cause avalanches on the city.\\
	\tabline& $\cdot$ Flame Mammoth occupies a factory and converts it for mass producing weapons needed for the war. However none of his subordinates follow him due his bad attitude as commander.\\
	\tabline& $\cdot$ Boomer Kuwanger captures the tower symbol of Abel city.\\
	\tabline& $\cdot$ Flame Stag disappears after defecting alongside Kuwanger.\\
	\tabline& $\cdot$ Spark Mandrill seizes the power plant, to redirect electricity onto Sigma's facility.\\
	\tabline& $\cdot$ Sting Chameleon occupies a front-line base set inside the forest.\\
	\tabline& $\cdot$ Sigma recalls Overdrive Ostrich, praising him for his remaining abilities. Ostrich pledge his loyalty to Sigma, but remains in the rears.\\
	\tabdot& Sigma sets up his fortress where he leads the operations.\\
	\tabline& $\cdot$ Scientists create Bospider to be a large-scale crusher with an invincible body, but a mistake causes its core to expose after a shock. Sigma decides to use it as guardian to his fortress.\\
	\tabline& $\cdot$ Rangda Bangda is created to crush enemies who enter Sigma's fortress.\\
	\tabline& $\cdot$ Scientists begin the construction of a giant dinosaur-based mechaniloid and a wolf-based mechaniloid capable to combine with Sigma to further enhance his power.\\
	\tabdot& Zero is appointed as leader of the 17th Elite unit and of the entire Haunter organization, being the highest in rank remained to oppose Sigma.\\
	%
	% X1
	%
	\midrule
	\rowcolor{Aquamarine}
	\multicolumn{2}{c}{\textbf{Mega Man X}}\\
	\addlinespace[1.5ex]
	\tabdot& Sigma's troops launch an attack on Abel city's highway. X intervenes to stop the fight.\\
	\tabline& $\cdot$ Vile attacks X on the highway, but Zero's intervention forces him to retreat. In the battle his ride armor is damaged.\\
	\tabdot& Zero entrusts X with dealing with the eight mavericks. Meanwhile he starts searching for Sigma's hideout.\\
	\tabdot& Magna Centipede is captured by the Mavericks, and brainwashed into a loyal soldier.\\
	\tabdot& X manages to defeat all eight mavericks that were threatening Abel city. In the meanwhile he also find capsules from which Dr.~Light AI gives him first armor pieces, now completed.\\
	\tabdot& Vile repairs and upgrades his ride armor, while also setting up a plan to take down both Zero and X.\\
	\tabline& $\cdot$ Sigma recovers defeated mavericks' dead bodies and resurrect them as an additional line of defense for his fortress.\\
	\tabdot& Zero finds Sigma's hideout, and ask X to reach him to begin the final assault.\\
	\tabdot& X and Zero enter Sigma's fortress but are ambushed by Vile, who captures both of them. However both Zero and X manage to break free from their prison and to take down Vile. Zero, however, dies in the fight too.\\
	\tabdot& X continues traveling through the fortress. Scientists, worried, accelerate the development of the tyrannosaurus mechaniloid mounting its head onto a moving platform, thus creating D-REX, which is however destroyed by X.\\
	%	\tabdot& Sigma manages to separate his mind from the body. His mind flee, while his original body remains back to fight X.\\
	\tabdot& X finally reaches Sigma, and the two fight. X defeats Velguarder first and Sigma immediately after. Sigma decides then to combine himself with the still incomplete wolf-mechaniloid built for him and become Wolf Sigma. X, however, manages to destroy him again. After Sigma's defeat, his fortress begins to collapse and falls into the ocean. X teleports outside.\\
	\tabdot& Sigma's mind separates from the body, surviving the clash with X.\\
	\midrule
	
	%
	% X1 - X2
	%
	\rowcolor{Aquamarine}
	\multicolumn{2}{c}{\textbf{Between X1 and X2}}\\
	\addlinespace[1.5ex]
	\timepoint{X returns the First Armor to Dr.~Light. The data acquired from the fights are used by Light's AI to upgrade the armor.}
	\tabdot& X is appointed leader of the 17th Elite Unit.\\
	\tabdot& Zero's control circuit is retrieved by the Maverick Hunters form the remaining of Sigma's fortress\\
	\tabdot& The X-Hunters, led by Serges, emerge from the shadows and take control of Sigma's army.\\
	\timepoint{Zero's body remains are retrieved by the maverick forces. Serges begins its reconstruction.}
	\tabdot& To increase the size of their army, the X-Hunters set up different factories to mass-produce Mavericks.\\
	\tabline& In one of these factories, Wire Sponge is created. Despite his childish attitude, he is enrolled in the mavericks army due his strength.\\
	\tabdot& The X-Hunters retrieve the remains of Zero's body and Serges begins its reparations.\\
	\tabdot& Crystal Snail, a mysterious reploid, is enrolled in Sigma's army.\\
	\tabline& $\cdot$ Wheel Gator reaches the Maverick army, hoping to satisfy his destructive impulses.\\
	\tabdot& Serges begins to build a new body for Sigma.\\
	\tabdot& The battle between Maverick Hunter and Mavericks continues, with heavy losses from both sides.\\
	\tabdot& Several Maverick Hunters bases are attacked and destroyed. The total number of effective in the Maverick Hunters organization is reduced to a quarter of its original value.\\
	\tabdot& Studying the defeated mavericks, Dr.~Cain discovers a chip in reploids brain which causes them to go maverick. The chip bears Sigma's insignia.\\
	\timepoint{Zero's reconstruction is almost finished. However his control circuit is still far from completion.}
	\tabdot& After some investigations, the Maverick Hunters locate the factory responsible for creating mavericks. X and his unit are dispatched to destroy the facility. In total six months have been passed since the end of the first uprising.\\
	%%
	% X2
	%%
	\rowcolor{Aquamarine}
	\multicolumn{2}{c}{\textbf{Mega Man X2}}\\
	\addlinespace[1.5ex]
	\tabdot & X destroys the maverick factory, where an army of giant mechaniloid CF-0 were being constructed.\\
	\tabdot & The destruction of the facility slows down X-Hunter's plan. To mitigate such problem, the X-Hunters releases their eight SA-class Mavericks to wreak havoc, hoping to gain enough time to complete their schemes.\\
	\tabline& $\cdot$ Wire Sponge is released inside the Weather Control facility, where he starts playing around with its controls, changing the nearby weather.\\
	\timemoment{Morph Moth is deployed in the scrapyard, to create new mavericks out of scrap.}
	\timemoment{Flame Stag reappears, and conquers a volcano aiming to cause an eruption to cover the sky and begin a new ice age.}
	\timemoment{Magna Centipede conquers the central computer, and from there begins to spread the Maverick Virus around the world.}
	\timemoment{Overdrive Ostrich conquers an abandoned missile base in the desert, aiming to use the remaining missiles to destroy the Maverick Hunters HQ.}
	\timemoment{Bubble Crab is dispatched to defend an underwater base used to ship mavericks around the world.}
	\timemoment{Wheel Gator leads the giant Dinosaur Tank aircraft carrier into a city, to wipe it out.}
	\timemoment{Crystal Snail is sent into the energen mines, to dig out crystals used to generate energy.}
	\timepoint{X destroys two of the SA-class Maverick quicker than expected. The three X-Hunters are forced to take action challenging X to a duel, using Zero's repaired parts as bait.}
	\timepoint{X fights against the three X-Hunters and wins, delivering Zero's parts to Dr.~Cain which immediately begins his reconstruction.}
	\timepoint{Deprived of their main weapon, Serges hurry in completing Sigma's body. To mitigate the loss of Zero, he also creates a replica of him, loyal to Sigma and with a black armor.}
	\timepoint{X manages to stop all remaining mavericks. In the meanwhile, he also obtains the Second Armor from Dr.~Light capsules.}
	\timepoint{Dr.~Cain manages to locate the X-Hunters fortress, situated in the north pole.}
	\timepoint{Serges completes both Sigma's body and fake Zero.}
	\timemoment{Sigma learns from Serges about Zero's true origins.}
	\timepoint{X storms the X-Hunters base. The three X-Hunters try to stop him using all their available tools, but are destroyed}
	\timemoment{After X-Hunter's defeat, Sigma reveals himself to X, and gives him an appointment to the Central Computer. After the message the base explodes.}
	\timepoint{X meet with Sigma at the Central Computer. Sigma uses the Zero replica to attack X, but the real Zero arrives and destroys his copy.}
	\timemoment{Zero proceeds to destroy the main computer.}
	\timemoment{X and Sigma clashes. X destroys Sigma's new body.}
	\timemoment{Sigma assumes his true form of virus, and materializes to try killing X. X however manages to defeat him and force Sigma to flee.}
\end{tabularx}


\chapter{Damage Tables}
%Damage table.tex
Here are reported damage tables for every weapon and every main bosses per game. For each sub:weapon two values are reported, standing for regular and charged shots (except sub:weapons which don't deal damages), and values refer to damage per single hit, not counting eventual multiple hits. For the X:buster four values are reported, one for each level charge included the one given by the arm parts.
%\begin{landscape}
\section{X1}
Damage chart provided by the \emph{Mega Man Knowledge Database}~\cite{wiki:damage_chart_X1}

\begin{table}[htp]
	\resizebox{\columnwidth}{!}{
	\begin{tabular}{r *{10}{c}}
		\toprule
		\huge{\textbf{Boss}}&{\includegraphics[width=30px, height=25px]{figures/X1/weapons/Xbuster.png}} & {\includegraphics[width=30px, height=25px]{figures/X1/weapons/Homig_T.jpg}} &
		{\includegraphics[width=30px, height=25px]{figures/X1/weapons/C_sting.jpg}} &{\includegraphics[width=30px, height=25px]{figures/X1/weapons/Rolling_S.jpg}} &
		{\includegraphics[width=30px, height=25px]{figures/X1/weapons/F_wave.jpg}} &{\includegraphics[width=30px, height=25px]{figures/X1/weapons/Storm_T.jpg}} 
		&{\includegraphics[width=30px, height=25px]{figures/X1/weapons/E_Spark.jpg}} &{\includegraphics[width=30px, height=25px]{figures/X1/weapons/B_cutter.jpg}} 
		&{\includegraphics[width=30px, height=25px]{figures/X1/weapons/S_ice.jpg}} &{\includegraphics[width=30px, height=25px]{figures/X1/weapons/hadoken_sprite.png}}\\
		\midrule
		%name				X-buster   HT   CS  RS   FW   ST   ES   BC   SI   H
		Launch Octopus&	 	1:2:3:3& 1:2& 1 & 3:4& 0:0& 1:2& 1:2& 1:2& 1:2& 32\\
		Chill Penguin&	 	1:2:3:3& 1:2& 1 & 1:2& 3:4& 1:2& 1:2& 1:2& 1:2& 32\\
		Armored Armadillo& 	1:1:1:1& 1:2& 1 & 1:2& 0:2& 0:2& 3:6& 1:2& 1:2& 32\\
		Flame Mammoth& 		1:1:2:2& 1:2& 1 & 1:2& 1:2& 3:4& 1:2& 1:2& 1:2& 32\\
		Storm Eagle& 		1:1:2:2& 1:2& 3 & 1:2& 1:2& 1:2& 1:2& 1:2& 1:2& 32\\
		Boomer Kuwanger&	1:2:3:3& 3:4& 1 & 1:2& 1:2& 1:2& 1:2& 1:2& 1:2& 32\\
		Spark Mandrill&		1:2:3:3& 1:2& 1 & 1:2& 1:2& 1:2& 1:2& 1:2& 3:4& 32\\
		Sting Chameleon&	1:1:2:2& 1:2& 1 & 1:2& 1:2& 1:2& 1:2& 3:4& 1:2& 32\\
		Vile &				1:2:4:4& 3:3& 2 & 4:4& 1:1& 1:4& 2:6& 2:6& 2:8& 32\\
		Bospider&			1:2:3:3& 1:2& 1 & 1:2& 1:2& 1:2& 1:2& 1:2& 3:4& 32\\
		Rangda Bangda&		1:1:2:2& 1:2& 3 & 1:2& 1:2& 1:2& 1:2& 1:2& 1:2& 11 per eye\\
		D:Rex&				1:1:2:2& 1:2& 1 & 1:2& 1:2& 1:2& 1:2& 3:4& 1:2& 32\\
		Velguarder&			1:2:3:3& 1:2& 1 & 1:2& 1:2& 1:2& 1:2& 1:2& 3:4& 32\\
		Sigma&				1:1:1:1& 1:1& 1 & 1:1& 1:1& 1:1& 2:3& 1:1& 1:1& 32\\
		Wolf Sigma&			0:0:0:1& 0:0& 0 & 2:2& 0:0& 0:0& 0:0& 0:0& 0:0& 0\\
		\bottomrule
	\end{tabular}
	}	
	\caption{Damage chart for main bosses in Mega Man X1.}
\end{table}
%\end{landscape}

\newpage
\section{X2}
Damage chart provided by the \emph{Mega Man Knowledge Database}~\cite{wiki:damage_chart_X2}. 
Please not that:
\begin{itemize}
\item For the X-buster, the fourth value is the damage dealt only by the second shot when fired at full power.
\item For the Shoryuken, damage reported is applied every two frame of contact, the first number represent the damage dealt to a boss without invincibility frame (i.e the first hit) and the second the damage dealt trough invincibility frames (check sec.~\ref{shoryuken} for detail on how damage dealt is calculated).

\item For Silk Shot, the first set of data is related to the boss' original stage, while the second one refers to the boss rematch, which always results in scrap metal to be tossed. The only exception are the X-Hunters, that can be possibly faced in each stage and hence are subjected to each variant of the silk shot. In this case only the main weakness is reported. However this does not apply for their rematches, as the arena is fixed just like the other bosses.% values follow the following list: crystals $\rightarrow$ leafs $\rightarrow$ rocks $\rightarrow$ scraps.
\end{itemize}
%\begin{landscape}	

\begin{table}[htp]
	\resizebox{\columnwidth}{!}{
		\begin{tabular}{r *{11}{c}}
			\toprule
			\huge{\textbf{Boss}}& {\includegraphics[width=30px,height=25px]{figures/X1/weapons/Xbuster.png}} &{\includegraphics[width=30px, height=25px]{figures/X2/weapons/C_hunter.png}}
			&{\includegraphics[width=30px, height=25px]{figures/X2/weapons/B_splash.png}} &{\includegraphics[width=30px, height=25px]{figures/X2/weapons/S_shot.png}} 
			&{\includegraphics[width=30px, height=25px]{figures/X2/weapons/S_wheel.png}} &{\includegraphics[width=30px, height=25px]{figures/X2/weapons/S_slicer.png}} 
			&{\includegraphics[width=30px, height=25px]{figures/X2/weapons/S_chain.png}} &{\includegraphics[width=30px, height=25px]{figures/X2/weapons/M_mine.png}} 
			&{\includegraphics[width=30px, height=25px]{figures/X2/weapons/S_burner.png}} &{\includegraphics[width=30px, height=25px]{figures/X2/weapons/G_crush.png}}
			&{\includegraphics[height=25px]{figures/X2/weapons/Shoryuken_ico.png}}\\
			\midrule
			%name				X-buster     CH   BS  	SSh   	  SW  SS  SC  MM  SB  GC S
\small{Giant Mechaniloid CF 0} & 3:5:8:	&- 	&- 	&-	&-	&- 	&- 	&- 	 &-	&- 	&- \\
			Wire Sponge 	    & 1:1:2:4   &0:0&1:1&1-1:1-2&1:1&2:5&1:1&1:2&2:1& 2  &16:8\\
			Morph Moth			& 1:1:2:4	&0:0&1:1&1:1  &1:1&1:1&1:1&1:2&3:6& 2	  &16:8\\
			Flame Stag			  & 1:1:2:4	&0:0&2:2&1-1:1-1	&1:1&2:2&1:1&1:1&1:1& 2-3 &8:8\\
			Magna Centipede		  & 1:1:2:4	&0:0&1:1&2:4 		&1:1&1:1&2:2&1:1&1:1& 2   &16:8\\
			Overdrive Ostrich	  & 1:1:2:4 &3:0&1:1&2-2:2-2    &1:1&1:1&1:1&1:1&1:1&2    &16:8 \\
			Bubble Crab			  & 1:1:2:4 &0:0&1:1&1-1:1-1    &3:4&1:1&1:1&1:1&1:1&2 	  &16:8 \\
			Wheel Gator			  & 1:1:2:4	&0:0&2:1&1:1        &1:1&1:1&3(5)\footnotemark:5&1:1&1:1&2&16:8 \\
			Crystal Snail		  & 1:1:2:4	&0:0&1:1&1-1:1-1	&1:2&1:1&1:1&3:4&1:1&2 	&16:8 \\
			Violen				  & 1:1:2:4 &0:0&2:2&\makecell[ct]{3:5(leaf)\\1:1 (other)}&1:1&2:2&1:1&1:1&1:1&2&-\\
			Neo:Violen		  & 1:1:2:3 &0:0&2:4&0:0&0:0&0:0&0:0&0:0&0:0&2&-\\
			Serges&1:1:2:4	  & 0:0	&1:1&\makecell[ct]{1:2(rocks)\\3:5(crystal)\\1:1 (other)} 	& 1:1	&2:5&1:1&1:2&2:1&2 	&- \\
			\makecell[rt]{Serges Tank\\(main body)}&1:1:2:3	&0:0&1:1&1:2&0:1&2:5&1:1&1:1&2:1&2 	&- \\
			Agile& 		1:1:2:4&0:0	&1:1&\makecell[ct]{3:5(rocks)\\1:1 (other)} 	&1:2 		&1:1 	&1:1 	&3:4 	&1:1 		&2 	&-\\
			Agile Flyer&1:1:2:3	&0:0&0:0&0:0&0:0&0:0&0:0&2:2&0:0&2 	&16:8 \\
			Zero&0:1:2:3&0:0&0:0&0:0&0:0&0:0&0:0&0:0&2:1&0&16:8\\
			Neo:Sigma& 1:2:2:4&0:0&0:0&0:0&0:0&2:4&0:0&0:0&0:0&3&16:8 \\
			Sigma Virus& 0:0:1:3&0:0&0:0&0:0&0:0&0:0&2:1&0:0&0:0&0:0&16:8\\
			\bottomrule
		\end{tabular}
	}	
	\caption{Damage chart for main bosses in Mega Man X2.}
\end{table}
\footnotetext{If Wheel Gator is hit by the Strike Chain when flashing, the damage dealt will count as 5 instead of 3}
%\end{landscape}

\newpage
\section{X3}
Damage chart provided by the \emph{Mega Man Knowledge Database}~\cite{wiki:damage_chart_X3}. 
Please not that:
\begin{itemize}
	\item For the X-buster, the fourth value is the damage dealt by the combined shot.
	\item For normal \emph{Acid Burst}, the first value refers to the glob itself, while the second indicates damage dealt by the droplets.
	\item For normal \emph{Triad Thunder}, the first value refers to the electric pods, while the second for the bolts. For the charged version, first value indicate damage dealt by the shockwave, while the second for  the electric balls.
	\item For charged \emph{Ray Spalsher}, the first value refers to the damage dealt by the pod itself, while the second for the beam it shots.
	\item For charged \emph{Frost Shield}, the first value refers to the damage dealt by the spikes when mounted on the X-buster, while the second for when they detach from it.
	\item For the \emph{Z:Saber}, the first value refers to the damage dealt at close range, while the second for the shockwave plus the three slashes that follow it.
\end{itemize}
%\begin{landscape}	

\begin{table}[htp] %TODO: controllo danni Vile tra le 2 battaglie
	\resizebox{\columnwidth}{!}{
		\begin{tabular}{r *{10}{c}}
			\toprule
			\huge{\textbf{Boss}}& {\includegraphics[width=30px,height=25px]{figures/X1/weapons/Xbuster.png}} &{\includegraphics[width=30px, height=25px]{figures/X3/weapons/A_burst.jpg}}
			&{\includegraphics[width=30px, height=25px]{figures/X3/weapons/P_bomb.jpg}} &{\includegraphics[width=30px, height=25px]{figures/X3/weapons/T_thunder.jpg}} 
			&{\includegraphics[width=30px, height=25px]{figures/X3/weapons/S_blade.jpg}} &{\includegraphics[width=30px, height=25px]{figures/X3/weapons/R_splasher.jpg}} 
			&{\includegraphics[width=30px, height=25px]{figures/X3/weapons/G_well.jpg}} &{\includegraphics[width=30px, height=25px]{figures/X3/weapons/F_shield.jpg}} 
			&{\includegraphics[width=30px, height=25px]{figures/X3/weapons/T_fang.jpg}} 
			&{\includegraphics[height=25px]{figures/X3/weapons/Saber.png}}\\
			\midrule
			%name		   	X-buster 	AB   PB    	TT   	  SB   RS     GW   FS   TF   ZS
			Maoh the Giant&	 2:3:4:-& -& - & -& -& -&-& -& -& -\\
			Blast Hornet&	 1:1:2:3& 1-1:2& 1:1 &  1-1:1-1& 1:1& 1:1-1& 4:6& 1:1-1& 0:0& 16:20\\
		Blizzard Buffalo &	 1:1:2:3& 1-1:1& 3:3 &  1-1:2-1& 1:1& 0:1-0& 0:0& 1:1-1& 0:0& 16:20\\
		Gravity Beetle &	 1:1:2:3& 1-1:1& 2:2 &  1-1:1-1& 1:1& 3:5-0& 0:0& 1:1-1& 0:0& 16:20\\
		Toxic Seahorse &	 1:1:2:3& 1-1:1& 1:1 &  1-1:1-2& 1:1& 1:1-1& 0:0& 3:5-1& 0:0& 16:20\\
		Volt Catfish	 &	 1:1:2:3& 1-1:2& 1:1 &  0-0:1-0& 1:2& 1:1-1& 0:0& 1:1-1& 2:3& 16:20\\
		Crush Crawfish	 &	 1:1:2:3& 1-1:1& 0:0 &  3-2:3-3& 0:0& 1:1-1& 0:0& 1:2-1& 0:0& 16:20\\
		Tunnel Rhino	 &	 1:1:2:3& 3-2:5& 0:1 &  0-0:2-0& 1:1& 1:2-1& 0:0& 1:1-1& 0:0& 16:20\\
		Neon Tiger		 &	 1:1:2:3& 1-1:1& 1:1 &  1-1:2-1& 2:4& 0:0-0& 0:0& 1:2-1& 0:0& 16:20\\
		Bit				 &	 1:1:2:3& 1-1:1& 1:1 &  3-2:1-5& 1:1& 1:1-1& 0:0& 3:5-3& 0:0& -\\
		Byte			 &	 1:1:2:3& 1-1:1& 1:1 &  1-1:1-1& 1:1& 2:5-2& 0:0& 1:1-1& 2:5& -\\
GodKarmachine O Inary 	 &	 1:1:2:3& 1-1:1& 2:2 &  1-1:1-1& 1:1& 3:5-3& 0:0& 1:1-1& 0:0& -\\
Press Disposer		 	 &	 1:1:2:3& 1-1:1& 1:1 &  1-1:1-1& 1:1& 2:5-2& 0:0& 1:1-1& 2:5& -\\
	Vile mk.II - DRA 00	 &	 1:1:2:3& 1-1:1& 1:1 &  0-0:1-0& 2:5& 2:5-2& 0:0& 1:1-1& 0:0& -\\
Vile mk.II - Goliath	 &	 1:1:2:3& 1-1:1& 2:3 &  1-1:0-2& 1:1& 1:1-1& 0:0& 1:1-1& 0:0& -\\
			Vile mk.II   &	 1:1:2:3& 1-1:1& 1:1 &  0-0:1-0& 2:5& 2:5-2& 0:0& 1:1-1& 0:0& -\\
		Volt Kurageil    &	 1:1:2:3& 1-1:1& 1:1 &  3-2:-- & 1:1& 1:1-1& 0:0&  3:5-3& 0:0& 16:20\\
		Dr.~Doppler   	 &	 1:1:2:3& 3-1:5& 1:1 &  1-1:2-2& 1:1& 1:1-1& 0:0&  1:1-1& 1:1& 16:-\\
		Sigma		   	 &	 1:1:2:3& 1-1:1& 1:0 &  1-1:1-1& 2:5& 1:1-1& 0:0&  2:4-1& 0:0& 16:20\\
		Kaiser Sigma	 &	 0:1:2:3& 0-0:0& 0:0 &  0-0:0-0& 0:0& 0:0-0& 0:0&  0:0-0& 0:0& 16:20\\
		\bottomrule
		\end{tabular}
	}	
	\caption{Damage chart for main bosses in Mega Man X3.}
\end{table}


\bibliographystyle{IEEEtran} 
\bibliography{bibliography}

\end{document}

